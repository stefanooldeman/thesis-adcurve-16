\chapter{Planning} % de projectactiviteiten met een beschrijving van de onderlinge samenhang, de mijlpalen en een fasering in de tijd met een schatting van de te besteden uren voor de verschillende uit te voeren activiteiten;

\section{De opdracht} % de afstudeeropdracht met mogelijke deelopdrachten, met vermelding van de projectgrenzen en randvoorwaarden en, indien het afstudeerproject onderdeel uitmaakt van een groter project, de afbakening t.o.v. het grotere project;

\section{De projectorganisatie} % de projectorganisatie;


De software ontwikkeling valt onder leiding van M. Jorissen. In nauwe samenwerking met M. Gill zijn zij verantwoordelijk voor de project analyse en planning. Het development team bestaat uit zes ontwikkelaars waaronder één ontwerper D.  De architectuur wordt geleid door S. Pogrebnyak. Dit team is verantwoordelijk voor het uitwerken en oplossen van projecten. Hierbij word gebruik gemaakt van een aantal methodieken en gewoontes vanuit Agile-softwareontwikkeling, waaronder: Kanban, Daily standup, Retrospectives, Extreme Programming (XP), Test Driven Development (TDD) en Continuous integration (CI).

Ondanks dat er in de werkwijze vaak een uitgebreide planning of requirements analyse ontbreekt, word de productiviteit en software kwaliteit toch bewaakt door de principes uit XP zoals TDD en refactoring \footnote{“Code refactoring is het proces waarbij de structuur van bestaande code word gewijzigd zonder de functionaliteit te wijzigen”…“dit komt ten goede aan de onderhoudbaarheid van code, en creëert nieuwe architectuur of object modellen om code makkelijker te kunnen uitbreiden met nieuwe functionaliteiten“ \parencite{refactoring}}. Hierdoor kan het team zich veroorloven om alleen voor de huidige situatie en problemen te ontwerpen. Veel planning is hierdoor niet aanwezig en technische implementatie komt vooral tot stand door TDD, brainstorm sessies, en de vrijheid om te proberen en fouten te maken.

\subsection{Verantwoordelijkheden}

De student werkt sinds september 2014 bij shop2market en maakt onderdeel uit van het vaste development team. De voornaamste verantwoordelijkheid is om op innovatieve wijze software te ontwikkelen. Daarnaast wil de student zich graag specialiseren en heeft de ambitie om te werken aan projecten waarbij thema’s als data integraties, data analyses of infrastructuur een rol spelen.


\section{Activiteiten} % de projectactiviteiten met een beschrijving van de onderlinge samenhang, de mijlpalen en een fasering in de
tijd met een schatting van de te besteden uren voor de verschillende uit te voeren activiteiten;

\section{Risico's} % de eventuele risico’s met maatregelen;

\section{Oplevering}
% twee data voor het inleveren van concepten van de scriptie (circa 6 resp. 3 weken voor de inleverdatum van

de definitieve versie van de scriptie);
