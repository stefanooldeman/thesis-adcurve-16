\chapter{Inleiding}

In dit document wordt het onderzoek omschreven dat is uitgevoerd in opdracht van Shop2market.

Shop2market voorziet honderden webwinkels in de behoefte om winstgevend te kunnen adverteren. Omdat de belangrijkste functionaliteiten berusten op de beschikbaarheid van statistiek gegevens ligt dit proces aan de kern van de dienst.

De organisatie heeft in de loop der jaren verschillende oplossingen geprobeerd om te voorzien in deze ``data behoefte": van Mondrian tot en met verschillende NoSQL databases. Nu vier jaar later is het verwerken van de data te traag. En daarnaast zijn er nog veel beperkingen en problemen met de huidige oplossing.

Omdat er in het verleden verschillende oplossingen zijn geprobeerd, moet voorkomen worden dat een nieuwe oplossing na een half jaar opeens weer aan vervanging toe is. Daarom is gevraagd om technologieën te onderzoeken en een aantal Proof of Concept 's te ontwikkelen. Hierdoor moet duidelijk worden welke technologie er geschikt is om de huidige problematiek op te lossen.

In hoofdstuk \ref{ch:project} wordt het project toegelicht in de vorm van doelstellingen, type opdracht en scope hiervan. Vervolgens wordt het ontwerp van het onderzoek gepresenteerd door de onderzoeksvragen te omschrijven en de methodes die worden gebruikt tijdens het onderzoek.


In hoofdstuk \ref{ch:onderzoek} worden eerst de functionele en niet functionele eisen waaraan de oplossing moet voldoen omschreven in de MosCow prioriteiten lijst. Het tweede deel van dit hoofdstuk beschrijft de analyse van methodes en technologieën om statistieken te berekenen.


In hoofdstuk \ref{ch:resultaat} worden problemen in de huidige situatie onderzocht en wordt er advies gegeven over hoe deze te voorkomen zijn. Vervolgens wordt het onderzoek en het advies verwerkt in een experiment. Door middel van een Proof of Concept wordt duidelijk of de gevonden technologieën werkelijk toepasbaar zijn binnen de organisatie.

Als laatste worden er worden de oplossingen getoetst op volledigheid en wordt de oplossing verantwoord in \ref{sec:deelvraag5}.


