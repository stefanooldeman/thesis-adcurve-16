Er zijn een aantal projecten die ik heb willen doen binnen Shop2market en het onderwerp van mijn afstudeerproject was daar een van. Mijn interesse ligt voornamelijk in het implementeren van data flows. Ik heb eerdere ervaringen met data processing systemen als Vertica, Hadoop en Disco (MapReduce). Binnen Shop2market is de ervaring met databases echter op slechte voet afgelopen en er zijn veel problemen geweest met het inzetten van datawarehouse gerelateerde oplossingen.

Tijdens de oriënterende fase van dit project werd er gezocht naar een alternatief voor de huidige oplossing in MongoDB MapReduce. Daarbij kwam Golang al snel ter spraken. Maar het idee was om verschillende Proof of Concepts te ontwikkelen. Hierbij zou duidelijk worden welke oplossing het beste werkt.

In het kader van het onderzoek heb ik verschillende technologieën onderzocht. Tijdens het evalueren ontstond de situatie waarbij mensen al een voorkeur uitspraken. De twee gevonden tools Golang en Spark waren beide positief getest. Feitelijk gezien was de oplossing in Golang sneller. Maar in vergelijking is Spark een specifieke tool voor data processing en Golang een programmer taal. Ik heb verschillende betrokkenen uitgelegd dat het onderhouden van zelf ontwikkelde systemen kostbaar kan zijn.

Een developer zag meer problemen dan voordelen bij een oplossing in Spark. Voordat ik de resultaten zou presenteren heb ik daarom nogmaals om advies gevraagd aan de opdrachtgever. We waren het met elkaar eens: Het type beta mens zal op basis van feiten altijd achter de meest logisch keuze staan. Helaas was ik niet voorbereid en had ik geen feiten om mijn overtuiging te verklaren. Slechts de ervaring dat een kant-en-klaar oplossing minder onderhoud oplevert. Toch heb ik de presentatie gemaakt waarin de feiten uit het onderzoek zijn gepresenteerd. Daarnaast heb ik de voor en nadelen uitgelegd voor beide oplossingen. Ik heb daarbij geprobeerd om het team te laten beslissen. Hoewel de argumenten overtuigend waren konden sommige mensen alleen vanuit een verdedigende houding reageren.

Dit resulteerden in een situatie waarin er als groep geen beslissing werd genomen. De discussie zette zich voort; de vraagtekens die mensen hadden kon ik eenvoudig beantwoorden met het onderzoek. Dit hielp niet en de groep kwam niet tot consensus. Uiteindelijk ben ik terug gevallen op de feiten en heb zelf een conclusie uitgesproken. Hierdoor werd de snelste inderdaad de gekozen oplossing.

Uiteindelijk ben ik ook enorm tevreden met het resultaat. De verdere ontwikkeling van het project is erg interessant en technisch leerzaam. Toch heb ik veel geleerd van de situatie. Doordat ik objectief wilde blijven kwam ik automatisch in een adviesrol terecht. Ik voelde mij niet comfortabel in die positie en hou niet van politiek binnen een organisatie.
Maar met de kennis van nu ben ik waarschijnlijk in staat om dit soort situaties te voorkomen. Ik herken nu dat de overtuigingen van mensen wellicht een sterkere rol speelt dan alleen de feiten.

Mijn eigenen objectiviteit bewaren was uiteindelijk nog het moeilijkste. Gelukkig heb ik veel onderzoek gedaan en ben ik er ook van overtuigd dat in de huidige context Golang de beste oplossing is. Dit was voor mij erg tegenstrijdig in het begin.

\ldots

\begin{flushright}
{\makeatletter\itshape
    \@author \\
    Utrecht, juni 2016
\makeatother}
\end{flushright}
\pagestyle{empty}

