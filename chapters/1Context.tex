\chapter{Context}
In dit hoofdstuk wordt beschreven bij welke organisatie de opdracht zich afspeelt. Wat waren de ontwikkelingen binnen de organisatie en wat zijn de redenen om een nieuw project te starten?

\section{Achtergrond}
\label{sec:achtergrond}

Shop2market is een software development bedrijf in de business-to-business sector. Het bedrijf heeft een missie om organisaties te helpen de winst uit online advertentiecampagnes te maximaliseren. Hiervoor is eerder een platform ontwikkeld dat vooral diende ter ondersteuning aan het adviesbedrijf. Een grote hoeveelheid van deze klanten waren webwinkels in het A segment. Maar omdat de integratie met het Shop2market platform en een webwinkel vaak maatwerk opleverde, duurde een integratie gemiddeld zes tot acht maanden. Hieruit valt ook te concluderen dat veel bedrijven niet de technologische middelen in huis hebben om zelfstandig te kunnen starten met adverteren.

Daarom werd in begin 2015 gestart met de ontwikkeling van een nieuwe dienst: Adcurve. Met alle ervaring vanuit de adviesorganisatie zijn veel processen vertaald naar functionaliteiten. Door de diverse functionaliteiten\footnote{ Denk hierbij aan datavisualiasties en beheeracties, soms ook wel "Actionable insights"\  genoemd.} in Adcurve kan de webwinkeleigenaar zijn online advertentiecampagnes controleren en binnen budget houden. Dit is mogelijk doordat Adcurve een partij is tussen de webwinkel en publishers. Met behulp van platforms zoals SEOShop of Magento is het mogelijk webwinkels binnen enkele minuten te integreren. 
Op basis van verzamelde gegevens zoals afkomstige bezoeken, bestellingen en advertentiekosten worden de nodige statistieken berekend. Met deze gegevens wordt de winstgevendheid per advertentie berekend.


Dit alles heeft als gevolg dat Shop2market op dit moment webwinkels in het midden en klein bedrijf  bedient, maar internationaal op een veel groter volume. Webwinkels zijn nu binnen enkele minuten geïnstalleerd en kunnen hun producten gemakkelijk adverteren via zogeheten publishers. Publishers zijn de bedrijven die de advertenties publiceren. Het soort advertenties verschilt nogal per publisher. Denk bijvoorbeeld aan ingekochte zoekresultaten, producten op prijsvergelijkers of affiliaties, maar ook producten op marktplaatsen. De gebruiker kan zelf publishers installeren binnen Adcurve zodat de benodigde integratie automatisch wordt afgehandeld. 

\clearpage

\section{Aanleiding} % de aanleiding tot de opdracht
\label{sec:aanleiding}

De afgelopen maanden zijn er vijf publishers geïntegreerd waarvoor een volledige data integratie is ontwikkeld. Bij deze publishers worden de in rekening gebrachten kosten door Adcurve geïmporteerd per advertentie, in plaats van berekend met een vast CPC (Cost Per Click) bedrag.
Omdat de belangrijkste functionaliteiten berusten op de beschikbaarheid van statistiek gegevens ligt dit proces aan de kern van de dienst. Het verwerken van de data bronnen verloopt niet altijd zonder fouten. Door de gewenste groei is het een prioriteit geworden om de datakwaliteit te kunnen waarborgen. In de huidige situatie is het nog lastig om te herstellen van fouten, doordat het berekenen van de statistieken een langzaam proces is.

Naast het garanderen van data kwaliteit is het belangrijk om functionaliteiten te verbeteren. De huidige strategie is om meer klanten aan te trekken door meer landen en industrieën te ondersteunen. Met de groei van het aantal publishers en data integraties is het probleem duidelijker geworden. De verbeterde oplossing moet helpen om internationale groei van Adcurve te ondersteunen en ruimte bieden om functionaliteiten betrouwbaarder en te maken. Er is hierdoor een toenemende wens ontstaan om de huidige oplossing te herzien.

\section{Organisatie} % beschrijving van de organisatie van de opdrachtgever en de plaats van de student daarin
\label{sec:organisatie}

Shop2market is met zijn team gevestigd in Hilversum en kent op dit moment achttien werknemers (zie figuur \ref{fig:orgchart}). De organisatie kan naar de theorie van
\autocite{mintzberg} worden omschreven als een Adhocracy: “Door de innovatieve aard van projecten is een organisatie gebaat bij flexibiliteit. Een formele hiërarchische structuur werkt daardoor minder goed.” Dit is herkenbaar en valt terug te leiden naar de professionele houding die van werknemers wordt verwacht. Er wordt autonomie gegeven om zelf structuur aan te brengen wanneer dit nodig is.

\begin{figure}[h]
    \includegraphics[width=0.9\textwidth]{organisation_structure.png}
    \caption{Organogram waarin het team en de relaties binnen Shop2market worden afgebeeld.}
    \label{fig:orgchart}
\end{figure}

\clearpage

\section{De kwestie} % een kwestie (aanleiding, het op te lossen probleem, de te vervullen behoefte of de te benutten kans);

Zoals te lezen valt in de aanleiding zijn er meerdere redenen voor dit project.

\begin{enumerate}
    \item Het kost momenteel te veel tijd om statistieken te berekenen. De berekeningen worden uitgevoerd met behulp van MongoDB MapReduce. Het team voorziet dat deze technologie niet genoeg schaalbaarheid biedt. Dit omdat rekentijd non-lineair is toegenomen in relatie tot de hoeveelheid data. Met de verwachte groei van Adcurve komen nieuwe eisen aan het licht en er moet naar een nieuwe oplossing worden gezocht.
    \item Bij voorkeur worden publishers geïntegreerd met behulp van API 's oftewel; een externe data bron. Op deze manier worden berekeningen uitgevoerd met precieze advertentiekosten. Maar doordat externe factoren nu een rol spelen in de berekeningen is het niet te garanderen dat de uitkomst altijd correct is. Zodra fouten intern of extern hersteld zijn, worden berekeningen voor een bepaalde dag, webwinkel of publisher opnieuw uitgevoerd. Het uitvoeren van dit soort correcties is tijdrovend door de huidige implementatie en gebruikte technieken.
    \item Webwinkels ontvangen tot soms tot 30 dagen na een bestelling een retournering van een of meerdere producten. Dit betekent dat een product niet verkocht is en de omzet uit de bestelling lager ligt dan is berekend. Het is wenselijk om berekende statistieken met betrekken tot retour bestellingen opnieuw te kunnen berekenen.
\end{enumerate}
