\chapter{Onderzoeksplan}
% de onderzoeksvragen, hoofdvraag met daaruit voortvloeiende deelvragen die moeten worden beantwoord

Welke oplossing moet er worden ontwikkeld zodat statistieken op tijd beschikbaar zijn en veranderingen in databronnen in korte tijd opnieuw geëvalueerd kunnen worden, zodat de stabiliteit en groei van Adcurve kan worden ondersteund?

Door middel van deelvragen word de garantie gegeven dat de hoofdvraag wordt beantwoord. Dit moet voldoende input geven om de ontwikkelfase van het project uit te voeren.

\begin{enumerate}
\item Ter verkenning van de eisen
    \begin{enumerate}
    \item Welke databronnen worden er gebruikt in het berekenen statistieken?

    \item Met welke frequentie zijn er veranderingen in de verschillende databronnen beschikbaar en voor welke databronnen moeten statistieken opnieuw worden geëvalueerd?

    \item Met welke groei van datasets moet worden rekening gehouden i.v.m. schaalbaarheid, om geplande groei van Adcurve te kunnen ondersteunen?

    \item Wat zijn de verwachtingen als het gaat om het “op tijd” of “in korte tijd” evalueren van databronnen en beschikbaar maken van statistieken?

    \item Van welke mogelijke risico’s en scenario’s m.b.t. de databronnen en infrastructuur moet er rekening gehouden worden om de stabiliteit van Adcurve te kunnen ondersteunen?

    \item Wat zijn de technische specificaties die worden gesteld aan een nieuwe te ontwikkelen statistieken platform?
    \end{enumerate}

\item Ter selectie van mogelijke oplossing
    \begin{enumerate}

    \item Welke externe projecten of technologieën komen in hun eigenschappen overeen met de gemaakte analyses?

    \item Hoe kunnen de beschikbare technologieën het beste worden ingezet om statistieken sneller en regelmatig te kunnen berekenen zodat functionaliteiten actueel blijven?

    \item Welke technologieën kunnen worden toegepast bij het ontwikkelde software ontwerp?

    \item Welke onbekende factoren moeten worden beantwoord met een of meerdere Proof of Concept(s) zodat een oplossing kan worden ontwikkeld?
    \end{enumerate}
\end{enumerate}


\section{Literatuur} %  (optioneel) een beschrijving van de belangrijkste literatuur die onderzocht zal worden

Data aggregaties, data transformaties en het selecteren en installeren van big data tools. Dit zijn slechts voorbeelden van concepten, kernbegrippen en mogelijke jargon die zullen voorkomen in het onderzoek of scriptie. Deze begrippen worden onderlegd door o.a. de volgende bronnen:

\begin{itemize}
    \item Data Mining, Concepts and Techniques \parencite{data-mining}
    \item UNIX for the impatient \parencite{unix}
    \item I <3 Logs, Event data, stream processing, and data integration \parencite{logs}
    \item Learning Storm \parencite{learning-storm}
    \item Fast Data Processing with Spark \parencite{spark}
    \item Real-Time Big Data Analytics \parencite{realtime-architectures}
    \item Hadoop The definitive guide, 4d edition \parencite{hadoop-definitive}
\end{itemize}

Tijdens de selectie van beschikbare technologieën wordt deels gebruik gemaakt van verschillende fases uit “de Berenschot-methode” \parencite{cuppen}.
Daarnaast zal tijdens het voeren van gesprekken, interviews en presentaties binnen de organisatie mogelijk gebruik worden gemaakt van de theorie uit “Adviseren als tweede beroep, resultaat bereiken als adviseur” \parencite{adviseren}.

\section{Resultaten} % de op te leveren producten met kwaliteitscriteria;

\section{Onderzoek methode} % de te gebruiken methoden/technieken/middelen (ook van het onderzoek) en, indien van toepassing, de

\section{Deelvragen} % deelvragen voorkomend uit de gekozen ontwerpmethode (optioneel);


