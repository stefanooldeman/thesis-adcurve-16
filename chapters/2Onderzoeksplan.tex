\chapter{Onderzoek}
% de onderzoeksvragen, hoofdvraag met daaruit voortvloeiende deelvragen die moeten worden beantwoord

In het onderzoeksplan worden een aantal vragen gesteld om de volgende hoofdvraag te beantwoorden \\

{\large \textit{"Hoe verzorgt een nieuwe implementatie voor het up-to-date houden van statistieken in Adcurve zodat gegevens altijd te verklaren zijn?"}} \\

\section{Onderzoeksvragen}

Om de hoofdvraag te beantwoorden zijn de volgende deelvragen geformuleerd

\begin{enumerate}
\item Wat zijn de functionele en niet functionele eisen waaraan de oplossing moet voldoen?
\item Wat zijn toonaangevende methodes en technologieën om statistieken te berekenen, die zowel passen bij de wensen en eisen van de opdracht?
\begin{enumerate}
    \item Hoe kan de opdracht worden opgelost?
\end{enumerate}
\item Welke scenario's komen met regelmaat voor waardoor gegevens niet te verklaren zijn?
\begin{enumerate}
    \item Wat zijn de gerelateerde technische factoren waardoor de scenario's voor verhindering zorgen in de huidige situatie?
    \item Wat zijn de mogelijke strategieën en technieken om dit op te lossen?
\end{enumerate}

\item Wat zijn de gepresenteerde oplossingen en waarom zijn deze volledig of niet?
\end{enumerate}

\begin{comment}
\section{Literatuur} %  (optioneel) een beschrijving van de belangrijkste literatuur die onderzocht zal worden

Voorbeelden van mogelijk jargon die zullen voorkomen in de thesis zijn data aggregaties, data transformaties en het selecteren en installeren van big data tools. Deze concepten worden onderlegd door onder andere de volgende literatuur:

\begin{itemize}
    \item Data Mining, Concepts and Techniques \parencite{data-mining}
    \item I <3 Logs, Event data, stream processing, and data integration \parencite{logs}
    \item Fast Data Processing with Spark \parencite{spark}
    \item Real-Time Big Data Analytics \parencite{realtime-architectures}
\end{itemize}

Tijdens de selectie wordt mogelijk gebruik gemaakt van verschillende fases uit “de Berenschot-methode” \parencite{cuppen}.
Daarnaast zal tijdens het voeren van gesprekken, interviews en presentaties binnen de organisatie mogelijk gebruik worden gemaakt van de theorie uit “Adviseren als tweede beroep, resultaat bereiken als adviseur” \parencite{adviseren}.
\end{comment}

\newpage
\section{Onderzoek methode} % de te gebruiken methoden/technieken/middelen (ook van het onderzoek) en, indien van toepassing, de
\label{sec:onderzoekmethode}

% \section{Deelvragen} deelvragen voorkomend uit de gekozen ontwerpmethode (optioneel);

Voor qualitatief onderzoek wordt een \textit{Case studie} gebruikt om de problemen in de gegeven context te analyseren. Methodes binnen dit type onderzoek zijn: explanatory, descriptive en exploratory. \parencite{john-dudovskiy}. Het onderzoek is ontworpen om de fases van een Case study uit te voeren. Het ontwerp is omschreven in tabel \ref{tab:onderzoekmethode}.

\begin{center}
\begin{table}[bh]
% \centering
\caption{Onderzoek methodes met te gebruiken methoden/technieken/middelen per deelvraag}
\label{tab:onderzoekmethode}
\def\arraystretch{1.5}
\begin{tabular}{|l|p{4cm}|p{2cm}|p{2.5cm}|p{4.5cm}|}
\hline
% \rowcolor{lightgray} 
\textbf{\#} & \textbf{Deelvraag} & \textbf{Type vraag} & \textbf{Methode} & \textbf{Actie / Resultaat} \\
\hline
1 & Wat zijn de functionele en niet functionele eisen, waaraan de oplossing moet voldoen?
  & Descriptive
  & Interviews
  & MosCow prioriteiten lijst en checklist samenstellen \\
\hline
2 & Wat zijn toonaangevende methodes en technologieën om statistieken te berekenen, en passen bij de wensen en eisen van de opdracht?
  & Descriptive
  & Literature-\newline research
  & Analyseren van bronnen m.b.v. van checklist wordt een shortlist samengesteld  \\
\hline
3 & Welke scenario's komen met regelmaat voor waardoor gegevens niet te verklaren zijn?
  & Descriptive
  & Interviews,\newline Literature-\newline research
  & Inventariseren op te lossen scenario 's met prioriteit d.m.v. Impact analysis \\
\hline
3a & Wat zijn de gerelateerde technische factoren waardoor de scenario's voor verhindering zorgen in de huidige situatie?
   & Descriptive
   & Literature-\newline research
   & Vergelijkingstabel huidige en wenselijke situatie met daarbij de technissche afhankelijkheden om een scenario te kunnen voorkomen \\
\hline
3b & Wat zijn de mogelijke strategieën en technieken om dit op te lossen?
   & Designing
   & Interviews,\newline Literature-\newline research
   & Door het toepassen van de vergelijkingstabel met gevonden technologieën worden mogelijke oplossingen ontworpen \\
\hline
4 & Wat zijn de mogelijke oplossingen en hoe wordt dit gevalideerd?
   & Comparative
   & Literature-\newline research,\newline group-discussion
   & Door het Analyseren van mogelijke ontwerpen en groep discussie worden er een beperkt aantal Proof of concept's geformuleerd met eisen. \\
\hline
5 & Wat zijn de gepresenteerde oplossingen en waarom zijn deze volledig of niet?
  & Explanatory
  & Case study
  & Conclusies op basis van van verzamelde gegevens tijdens Proof of concept- fase, zoals bijv. performance tests. Tabel van oplossingen met theorieën die het resultaat verklaren. \\
\hline
\end{tabular}
\end{table}
\end{center}