\chapter{Project}
% de onderzoeksvragen, hoofdvraag met daaruit voortvloeiende deelvragen die moeten worden beantwoord

In dit hoofdstuk worden het project kort toegelicht in de vorm van de doelstellingen, het type opdracht en de scope hiervan. Vervolgens wordt het ontwerp van het onderzoek gepresenteerd door de onderzoeksvragen te omschrijven en de methodes die worden gebruikt tijdens het onderzoek.

\section{Doelstelling} % de doelstellingen (wat moet na afloop van het afstudeerproject zijn bereikt);
\label{sec:doelstelling}

Webwinkel eigenaren moeten in staat zijn om beslissingen te maken op basis van correcte en actuele gegevens in Adcurve. Dit betekent dat de gegevens die Adcurve toont altijd te verklaren zijn en overeenkomen met de werkelijkheid. De data moet op tijd verwerkt zijn, en fouten moeten tijdig hersteld kunnen worden.

\section{Type opdracht}

Aan de hand van beschikbare technieken wordt er gekozen voor een aantal mogelijke oplossingen.\newline Deze oplossingen worden vervolgens getest middels een Proof of Concept (POC). Door dit onderzoek moet duidelijk gaan worden: Hoe kan de opdracht worden opgelost? Is dit mogelijk? En wat is er nodig om de oplossing naar productie te krijgen? Door de opdrachtgever is gevraagd om een aantal POC 's te ontwikkelen om tot een inzicht over de oplossing te komen. Er is hierdoor sprake van een ontwikkelopdracht.


\section{Scope}

De data aggregaties worden momenteel berekend op een aantal dimensies: per jaar, per maand, per week en per dag. De aggregaties worden in een aantal stappen uitgevoerd, in de volgende volgorde:
\begin{itemize}
\item per geadverteerd product bij een publisher.
\item totalen per publisher
\item totalen voor de gehele webwinkel
\end{itemize}
Het berekenen van statistieken op het laagste niveau geeft de grootste uitdaging. Dit komt mede omdat de invoer data (input) een groter volume heeft op dit niveau, in vergelijking tot latere aggregaties. Ieder POC beperkt zich daarom tot het berekenen van een aantal metrieken op het laagste aggregatie niveau voor webwinkels: geadverteerde producten.

\section{Onderzoek}

In \ref{sec:doelstelling} is de volgende doelstelling omschreven: ``Webwinkel eigenaren moeten in staat zijn om beslissingen te maken op basis van correcte en actuele gegevens in Adcurve. Dit betekent dat de gegevens die Adcurve toont altijd te verklaren zijn en overeenkomen met de werkelijkheid. De data moet op tijd verwerkt zijn, en fouten moeten tijdig hersteld kunnen worden.". 

Hieruit is een onderzoeksplan opgesteld. In het onderzoeksplan worden vijf deelvragen behandeld die samen de volgende 
hoofdvraag beantwoorden, namelijk:

\medskip
{\large \textit{"Hoe verzorgt een nieuwe implementatie voor het up-to-date houden van statistieken in Adcurve zodat gegevens altijd te verklaren zijn?"}}
\medskip

\subsection{Onderzoeksvragen}

Om de hoofdvraag te beantwoorden zijn de volgende deelvragen geformuleerd

\begin{enumerate}
\item Wat zijn de functionele en niet functionele eisen waaraan de oplossing moet voldoen?
\item Wat zijn toonaangevende methodes en technologieën om statistieken te berekenen, die passen bij de wensen en eisen van de opdracht?
\item Wat zijn de probleemscenario's waardoor gegevens niet te verklaren zijn en wat zijn mogelijke oplossingen?
\item Wat zijn de mogelijke oplossingen en hoe wordt dit gevalideerd?
\item Wat zijn de gepresenteerde oplossingen en waarom zijn deze volledig of niet?
\end{enumerate}


\newpage
\subsection{Onderzoek methode} % de te gebruiken methoden/technieken/middelen (ook van het onderzoek) en, indien van toepassing, de
\label{sec:onderzoekmethode}

% \section{Deelvragen} deelvragen voorkomend uit de gekozen ontwerpmethode (optioneel);

Voor kwalitatief onderzoek wordt een \textit{Case studie} gebruikt om de problemen in de gegeven context te analyseren. Methodes binnen dit type onderzoek zijn: explanatory, descriptive en exploratory. \parencite{john-dudovskiy}. Het onderzoek is ontworpen om de fases van een case studie uit te voeren. Het ontwerp is omschreven in tabel \ref{tab:onderzoekmethode}.

\begin{center}
\begin{table}[bh]
% \centering
\caption{Onderzoek methodes met te gebruiken methoden/technieken/middelen per deelvraag}
\label{tab:onderzoekmethode}
\def\arraystretch{1.5}
\begin{tabular}{|l|p{4cm}|p{2cm}|p{2.5cm}|p{4.5cm}|}

\hline
% \rowcolor{lightgray} 
\textbf{\#} & \textbf{Deelvraag} & \textbf{Type vraag} & \textbf{Methode} & \textbf{Actie / Resultaat} \\
\hline
1 & Wat zijn de functionele\newline en niet functionele eisen,\newline waaraan de oplossing\newline moet voldoen?
  & Descriptive
  & Interviews
  & MosCow prioriteiten lijst en checklist samenstellen \\
\hline
2 & Wat zijn toonaangevende\newline methodes en\newline technologieën om statistieken te berekenen, die passen bij de wensen en eisen van de opdracht?
  & Descriptive
  & Literature-\newline research
  & Analyseren van bronnen\newline m.b.v. van checklist wordt een shortlist samengesteld  \\
\hline
3 & Wat zijn de probleem\newline scenario's waardoor\newline gegevens niet te verklaren zijn en wat zijn mogelijke oplossingen?
  & Comparative
  & Interviews,\newline Literature-\newline research
  & Inventariseren op te lossen scenario 's\newline d.m.v. Impact analysis \\
  % 3a & Vergelijkingstabel huidige en wenselijke situatie met daarbij de technissche afhankelijkheden om een scenario te kunnen voorkomen \\
  % 3b & Door het toepassen van de vergelijkingstabel met gevonden technologieën worden mogelijke oplossingen ontworpen \\
\hline
4 & Wat zijn de mogelijke oplossingen en hoe wordt dit gevalideerd?
   & Designing
   & Literature-\newline research\newline group-\newline discussion
   & Door het Analyseren van\newline mogelijke ontwerpen en \newline groep- discussies worden\newline er POC 's\newline ontworpen met eisen. \\
\hline
5 & Wat zijn de gepresen-\newline teerde oplossingen\newline en waarom zijn deze\newline volledig of niet?
  & Explanatory
  & Case study
  & Conclusies op basis van verzamelde gegevens tijdens de proof of concept fase, zoals bijv. performance tests. Tabel van oplossingen met theorieën die het resultaat verklaren. \\
\hline
\end{tabular}
\end{table}
\end{center}

\clearpage

\section{Ethische overweging}

In het boek ``Business research methods"\ door \textcite{bryman2015business} worden ethische principes besproken om te valideren of een project op ethische wijze is uitgevoerd.

\begin{itemize}
    \item Is er schade toegebracht aan de deelnemers? Nee, omdat er sprake is van kwalitatief onderzoek zijn er geen deelnemers betrokken in dit onderzoek.
    \item Is er een gebrek aan toestemming of overeenstemming? Nee, De gebruikte data bronnen in dit onderzoek zijn verzameld voor het de ondersteuning van functionaliteiten van Shop2market en Adcurve. Hiervoor is instemming van alle klanten voor het gebruiken van de software door Shop2market. De data die mogelijk is verzameld over internetgebruikers is met instemming gebeurd middels de cookiewet.
    \item Is er sprake van een inbreuk van privacy? Nee, in de gebruikte data bronnen is geen aanwezigheid van persoonsgebonden informatie. Daarnaast zijn privacy gevoelige informatie zoals IP adressen en cookies niet aanwezig in de gebruikte data bronnen.
    \item Is er sprake geweest van bedrog? Dit is niet van toepassing omdat er geen deelnemers zijn betrokken in dit onderzoek.
    \item Als laatste verklaar ik dat deze thesis mijn eigen werk is. Tot mijn kennis is er geen vorm van copyright schending of zelfs plagiaat. Werken van andere zoals onderzoeken worden vermeld met gebruik van het APA referentie systeem.
\end{itemize}