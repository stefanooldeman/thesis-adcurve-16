\chapter{Onderzoek}
\label{ch:onderzoek}

Eerst worden de functionele en niet functionele eisen waaraan de oplossing moet voldoen omschreven in de MosCow prioriteiten lijst. Dit geeft antwoord op deelvraag 1 en is te lezen in \ref{sec:deelvraag1}.

In het tweede deel is te lezen in \ref{sec:deelvraag2}, en hier wordt onderzoek gedaan naar deelvraag 2: ``Wat zijn toonaangevende methodes en technologieën om statistieken te berekenen, die passen bij de wensen en eisen van de opdracht?". Dit beschrijft de analyse van methodes en technologieën om statistieken te berekenen.

Aan de hand van project eisen resulteert dit in een shortlist en wordt omschreven in \ref{sec:gevonden_tools}. Met behulp van een checklist en groepsdiscussie wordt geconcludeerd welke technologieën getest zullen worden middels een Proof of Concept.

\clearpage

% ACTIVITY Afleggen interviews rondom Requirements \& Constraints, documenteren use-cases
% RESULT MosCow prioriteiten lijst en checklist
\section{Functionele en niet functionele eisen}
\label{sec:deelvraag1}

Wat zijn de functionele en niet functionele eisen, waaraan de oplossing moet voldoen?

In een interview met Matthijs Jorissen zijn de functionele en niet functionele eisen besproken. Vervolgens is hierin een prioriteit aangebracht met behulp van de MoSCoW methode. Hierbij worden alle project eisen in vier groepen ingedeeld: priority groups “MUST have”, “SHOULD have”, “COULD have”, en “WON’T have” \textcite{ma2009effectiveness}.

\begin{comment}
Een functionele eis kan gezien worden als iets dat de gebruiker nodig heeft om het doel te bereiken of een bepaalde voorwaarde waaraan de oplossing moet voldoen.

Een non functionele eis is een beperking doe wordt opgelegd op een mogelijke oplossing, met het doel om functionele eisen te behalen of het doel van het project.
\end{comment}

\begin{table}[bh]
\centering
\caption{lijst van eisen geprioritiseerd met behulp van de MoSCoW analyse}
\label{table:requirements}
\def\arraystretch{1.5}

\begin{tabular}{|l|p{12.5cm}|}
\hline
\textbf{Prioriteit} & \textbf{Functionaliteit/Requirement}
\\ \hline
Must have           & Statistieken voor één webwinkel kunnen opnieuw worden gegenereerd zodat veranderingen in externe data bronnen, bijvoorbeeld orders of geïmporteerde kosten, opnieuw worden geaggregeerd. Dit moet kunnen tot 30 dagen terug.
\\ \hline
Must have           & Het creëren van data aggregaties voor alle webwinkels over een dag, mag niet meer tijd in beslag nemen dan de huidige oplossing nodig heeft. Dit is 1 uur en 30 minuten.
\\ \hline
Must have           & De kosten voor eventuele licenties en infrastructuur mogen niet hoger zijn dan dat voor de huidige oplossing.
\\ \hline
Should have         & Statistieken zijn altijd voor kantoor uren beschikbaar voor het dashboard en voor latere processen zoals het genereren van tips.
\\ \hline
Should have         & De gekozen oplossing heeft een productief programmeermodel en is afhankelijk van hardware architectuur.
\\ \hline
Could have         & De programmatuur is testbaar en hierdoor eenvoudig te onderhouden.
\\ \hline
Could have          & Data aggregaties voor een groep webwinkels kunnen op een ander tijdstip worden geaggregeerd i.v.m. verschillende tijdszones.
\\ \hline
Could have         & De nieuwe oplossing moet schaalbaar zijn tot 10.000 webshops.
\\ \hline
\end{tabular}
\end{table}

Een aantal gebruikte termen zoals simpel, snel etc. worden verduidelijkt met de bijbehorende definities die in overleg tot stand zijn gekomen:

\begin{itemize}
    \item \textbf{Een productief programmeermodel} in de context van dit project wordt omschreven in \textcite{asanovic2006landscape}: "programming models should be independent of the number of processors, they should allow programmers to use a richer set of data types and sizes, and they should support successful and well-known parallel models of parallelism"

    \item \textbf{Scalability} wordt door \textcite{dubey2005recognition} gedefinieerd als: Schaalbaarheid is het vermogen om de gewenste snelheid te bieden en te onderhouden al dan niet verbeteren. De industrie heeft hier twee manieren voor. ``Scale up"\ is de methode waarbij hardware wordt vervangen zodat betere prestatie geleverd kan worden. Bij ``scale out"\ wordt er extra hardware aangesloten zodat een bestaand systeem de toenemende werkdruk kan ondersteunen.
\end{itemize}

\begin{comment}
TODO
\begin{itemize}
    \item Data aggregaties moeten plaatsvinden zodra kosten beschikbaar zijn per publisher
    
    \item voor probleem scenario 1 en 4, moet mogelijk zijn om voor een individuele webwinkels de data te aggregeren zodat er correcties kunnen worden gemaakt bij data kwaliteit issues. 
    \item de snelheid waarmee data wordt verwerkt moet snel genoeg zijn om 30 dagen aan data binnen 1 dag te verwerken.
    
    \item De tijd die het kost voor data aggregaties moet voorspelbaar zijn, dit kan worden bereikt wanneer de oplossing lineair schaalbaar is.
\end{itemize}

\end{comment}




\clearpage

% ACTIVITY Onderzoeken van toonaangevende, passende technologieën
\section{Toonaangevende methodes en technologieën}
\label{sec:deelvraag2}
Wat zijn toonaangevende methodes en technologieën om statistieken te berekenen, die passen bij de wensen en eisen van de opdracht?

Uit verschillende websites zijn voorbeelden van technologieën gevonden voor het verwerken van data, zie tabel \ref{tab:sites}. Sommige van de technologieën zijn bekend binnen de organisatie en in overleg met de opdrachtgever zijn een aantal richtingen geselecteerd om te onderzoeken, namelijk: databases, distributed systemen, en programmeer talen die optimaal gebruik maken van hardware performance.

\begin{table}[h]
\caption{Voorbeelden van mogelijke technologieën}
\label{tab:sites}
\def\arraystretch{1.5}
\begin{tabular}{|l|p{12.5cm}|}
\hline
\textbf{Voorbeelden} & \textbf{Website}                                                                 \\ \hline
Databases     & https://github.com/onurakpolat/awesome-bigdata                                     \\ \hline
Hadoop                                  & https://experfy.com/blog/hadoop-market-size-adoption-growth-2020/                  \\ \hline
Spark                                   & http://datanami.com/2014/11/21/spark-just-passed-hadoop-popularity-web-heres/      \\ \hline
Hardware                  & http://radar.oreilly.com/2014/01/a-compelling-family-of-dsls-for-data-science.html \\ \hline
\end{tabular}
\end{table}

De huidige oplossing maak gebruik van een MapReduce algoritme. Doordat de bestaande oplossing wordt getest met behulp van een nieuwe technologie zal dit worden meegenomen in het onderzoek. Er zijn eerdere vermeldingen gevonden van MapReduce onder een andere naam, Monte Carlo is omschreven door \textcite{asanovic2006landscape} en eerdere referenties zijn gevonden in \textcite{lee2010debunking}. In het onderzoek zal ook duidelijk komen hoe verschillende technologieën hier anders mee om gaan.

\newpage

\subsection{Database oplossingen}
\label{sec:databases}
Een mogelijke oplossing is om een Datawarehouse te implementeren waarin alle benodigde data bronnen worden beheerd en data aggregaties worden uitgevoerd.

Voor het maken van data aggregaties zijn basis SQL functies nodig: SUM, MIN, MAX, AVERAGE, samen met de SQL clauses: GROUP BY en HAVING. \parencite{data-mining}
Omdat bijna alle database systemen deze functies ondersteunen is er gekeken naar databases die een hoog volume aan data kunnen verwerken.

Veel Database Management Systemen (DBMS), ondersteunen de benodigde functies maar implementeren transactionele mechanismes om data integriteit te garanderen, bijvoorbeeld: ``locking-based concurrency control". Dit introduceert significante overhead. \parencite{harizopoulos2008oltp}. Verder raakt een DBMS overbelast wanneer een gevraagde opdracht\footnote{ Ieder SQL statement wordt vertaalt naar een opdracht door middel van een query plan.} niet optimaal kan worden uitgevoerd. Zo moet de opgevraagde dataset in het geheugen (RAM) passen, of er wordt terug gevallen op de harde schijf (zogeheten ``Disk spills"). De gegeven use-case vraagt om het frequent uitvoeren van aggregaties en corrigeren van data bronnen. De combinatie hiervan introduceert mogelijke performance problemen. \parencite{kersten2011researcher}
%"Moreover, database queries often contain blocking operations that lead to a pipeline stall or spilling large intermediates back to the disks." \parencite{kersten2011researcher}

% \footnote{Door frequent te aggregeren en corrigeren van data bronnen door middel van incremental loading of vervangen van datasets door 'UPDATE' of 'DELETE and INSERT' statements}
% Het plan in onze use-case is om frequent data aggregaties uit te voeren zodat statestieken in adcurve worden gecorrigeerd.

In vergelijking tot eerder genoemde OLTP (Online Transaction Processing) databases uit \textcite{kersten2011researcher} zijn er genoeg OLAP (Online Analytical Processing) systemen beschikbaar voor Data Warehouse toepassingen \parencite{data-mining}.

% In de paper "How To Build a High-Performance Data Warehouse" worden door de auteurs drie verschillende architectuur storage engines omschreven: Shared Memory, Shared Disk en Shared Nothing.

Michael Stonebraker, database pionier en origineel auteur PostgreSQL
\footnote{
 Opvallend is dat Bijna alle High Performance en Analytical databases afstammen van Postgres \parencite{postgresforks}. Dit dient als getuige van de enorme hoeveelheid research die is geïnvesteerd in (opensource) oplossingen op basis van PostgeSQL.
}
, omschrijft drie verschillende database storage engines: shared memory, shared disk en shared nothing. ``Because shared nothing does not typically have nearly as severe bus or resource contention as shared-memory or shared-disk machines, shared nothing can be made to scale to hundreds or even thousands of machines. Because of this, it is generally regarded as the best-scaling architecture". \parencite{dewitt2006build}


\subsubsection{\textbf{Conclusie}}

Er zijn verschillende beschikbare technologieën gevonden in de vorm een DBMS  Er zijn verschillende databases gevonden waarvan uit onderzoek blijkt dat de databases schaalbaar zijn in te zetten. De volgende databases zijn hieruit omschreven: Terradata, GreenplumDb, Vertica. Doordat deze databases niet gebruik maken van traditionele mechanismes en gebruik maken van een Shared Nothing architectuur zijn deze schaalbaar in te zetten. Dit is genoeg motivatie om een POC uit te voeren een van de databases, namelijk GreenplumDb. Dit omdat hier geen licentie kosten aan verbonden zijn.

\clearpage

\subsection{Distributed oplossingen}
\label{sec:distributed}
De indruk binnen de organisatie is dat de huidige oplossing niet schaalbaar genoeg is. Binnen distributed systemen wordt schaalbaarheid gerealiseerd door de ``scale out"\ methode.

% MapReduce wordt gezien als de oplossing wanneer de grootte van de dataesest niet effectief verwerkt kunnen worden door steeds groter groeiende databases. (TODO reference this)

HP Vertica, werd al eerder besproken in \ref{sec:databases}, en wordt omschreven als een ``distributed database to mean a sharednothing, scale-out system" \parencite{lamb2012vertica}. Hoewel alle MapReduce platformen hun performance behalen op zogeheten ``Commodity hardware"\ wordt bij databases ``Modern hardware"\ aanbevolen. Daarnaast zijn databases veelal commercieel waarbij Hadoop veel open source distributies kent. \parencite{dean2008mapreduce}

Hadoop staat bekend als de technologie met grote clusters, een voorbeeld van een distributed oplossing en een implementatie van Map Reduce, zo schrijft \textcite{hadoop2013selection}: ``The open-source software framework known as Apache Hadoop has gained sizable acceptance in organizations, spurred on by the growing digital business appetite for big data". Echter weten veel organisaties niet hoe ze deze technologie effectief moeten inzetten. Zo valt te concluderen uit \textcite{hadoop2015adoption} dat adoptie van het Hadoop ecosysteem nog altijd nog altijd laag blijft.

Afgezien van de adoptie blijft Hadoop interessant. Doordat de grootste bedrijven Hadoop blijven inzetten, met een groot aantal gebruikers, blijft het platform zich sterk ontwikkelen. In de loop der tijd zijn er SQL talen zoals Hive ontwikkeld waardoor het gebruikersgemak vele malen eenvoudiger is geworden. Echter introduceert dit nieuwe uitdagingen. Doordat data sets frequenter worden opgevraagd is er mogelijke terugval in performance\footnote{Lees: adhoc query 's vs. reporting query 's}. \parencite{thusoo2010hive}

Daarnaast blijkt uit benchmarks door \textcite{armbrust2015spark} dat Spark SQL competitief is met vergelijkbare SQL talen op Hadoop zoals Hive en Impala\footnote{Impala en Hive worden aangeboden, afhankelijk van de Hadoop distributie. De performance is vergelijkbaar afhankelijk van de configuratie en hoeveelheid query 's. \parencite{hortonworks_benchmark}}.

Apache Spark, origineel ontwikkeld in UC Berkley heeft sinds zijn release in 2014 als open source project meer contributies gezien dan Apache Hadoop in totaal heeft ontvangen\footnote{Met over 400 contributies is Apache Spark het meest actieve data processing framework in de industrie}. \parencite{armbrust2015spark}. Dit is veelbelovend en ook vanuit Shop2market is er een enorme interesse. In 2015 zijn er al succesvolle experimenten uitgevoerd met SparkML voor een Machine Learning applicatie.

\subsubsection{\textbf{Conclusie}}

Er zijn verschillende technologieën gevonden die gebruik maken van een distributed of networked methode. Omdat in de huidige situatie de data sets niet groter zijn dan 2GB wordt er voor gekozen om geen POC uit te voeren met Hadoop.  In eerdere ervaring met Hive duren gegeven opdrachten minimaal tussen de 15 en 17 minuten. Dit is te verklaren doordat de SQL query 's worden vertaald naar de onderliggende MapReduce fases. \parencite{thusoo2010hive} Dit introduceert een significante overhead voor de huidige use case.

Daarnaast is gevonden dat Spark SQL deze overhead niet ervaart doordat alle data communicatie in-memory plaats vindt. In tegendeel tot de verschillende fases in Hadoop waarbij er wordt gelezen en geschreven van het filesysteem (HDFS). Dit is genoeg motivatie om een POC uit te voeren met Apache Spark
% waarbij Spark een vergelijkbaar algoritme toepast met dat van MapReduce maar veel performance gerelateerde issues voorkomen kunnen worden door het gebruik van in-memory data communicatie tussen verschillende fases.
% \verb TODO: add citation ``Comparing Apache Spark and Map Reduce with Performance Analysis using K-Means"

\clearpage

\subsection{Hardware oplossingen}
\label{sec:hardware}
Een van de oplossingen die wordt onderzocht is het gebruik maken van krachtige hardware componenten (zowel de CPU en de GPU). Dit is mogelijk met expert talen zoals; Verilog, CUDA of OpenCL of C++.


Op de universiteit van Stanford is hier veel onderzoek naar gedaan. Dit vereist alleen veel hardware specifieke kennis en moet in code worden uitgedrukt: "programming these devices to run efficiently and correctly is difficult, error-prone, and results in software that is harder to read and maintain." \parencite{sujeeth2011optiml}. Daarom is er een serie aan DSL's ontwikkeld\footnote{Voorbeelden van DLS 's zijn \TeX, HTML en SQL  \parencite{sigplan2000dsl}}

Terwijl \textcite{sujeeth2011optiml} een veelbelovende oplossing lijkt aan te bieden is het project nog altijd in Alpha versie \parencite{optiml_project_home}. Hetzelfde team presenteerde in 2014 Forge, een DSL voor expert talen met veelbelovende test resultaten: "Forge-generated Delite DSLs perform within 2x of hand-optimized C++ and up to 40x better than Spark" \parencite{sujeeth2014forge}. Helaas lijkt hierbij hetzelfde probleem te spelen en is er niet genoeg documentatie gevonden om hier een POC mee te starten.

Binnen Shop2market zijn veel performance gerelateerde projecten opgelost met Golang. Naast dat dit erg productief is bevonden biedt de moderne taal een alternatief concurrency model. "C, C++, and to some extent Java are quite old, designed before the advent of multicore machines, networking, and web application development. There are features of the modern world that are better met by newer approaches, such as built-in concurrency." \parencite{pike2012go}.

\subsubsection{\textbf{Conclusie}}

De besproken technologieën zijn de DSL talen: OptiML, OptiSQL en Forge. Bij gebrek aan praktische voorbeelden en algemene documentatie is er voor gekozen om hier niet verder in te investeren. Er zou niet voldoende kennis zijn om een mogelijke POC naar productie te brengen. Hoewel Forge tijdens de benchmarks extreem goed presteert, wordt niet vermeld welke algoritmes hierbij zijn gebruikt. Omdat in de huidige implementatie gebruik wordt maakt van MapReduce (de Monte Carlo methode) hoeft deze performance niet behaalt te worden. Zo schrijft \textcite{lee2010debunking}: "Monte Carlo algorithms are generally compute-bound with regular access patterns, which makes it a very good fit for SIMD architectures", in vergelijking met een GPU processor. Golang biedt hier een mogelijke oplossing, omdat de taal een productieve manier biedt om optimaal gebruik te maken van Multicore CPU's.

\clearpage

\section{Conclusies}
\label{sec:gevonden_tools}
Uit de verschillende oplossingsrichtingen zijn een aantal technologieën onderzocht in sectie \ref{sec:deelvraag2} t/m \ref{sec:hardware}. Deze zijn besproken in een groepsdiscussie waarbij de volgende non functionele eisen worden getoetst. Hiervoor is een checklist samengesteld, gebaseerd op de projecteisen omschreven in \ref{sec:deelvraag1}, en aanvullende eisen (1 en 5) die ter spraken zijn gekomen tijdens de discussie. Per vergelijkingsmatrix is af te lezen hoe iedere technologie scoort tegenover de volgende checklist in figuur \ref{tab:checklist}.

\begin{table}[h]
\centering
\caption{Checklist met criteria gebruikt in vergelijkingsmatrix}
\label{tab:checklist}
\def\arraystretch{1.2}
\begin{tabular}{l|p{11cm}c}
label       & criteria                                                          & zwaarte \\ \hline
support     & genoeg ondersteuning vanuit open source of commerciële industrie; & 3       \\
scalability & indicatoren tot schaalbaarheid en high performance;               & 2       \\
productief  & ondersteund een productief programmeer model;                     & 2       \\
testbaar    & code is testbaar door middel van unit tests.                      & 1       \\
kennis      & er is genoeg kennis beschikbaar binnen of buiten de organisatie   & 1      
\end{tabular}
\end{table}

\subsubsection{\textbf{Database oplossingen}}

In vergelijkingsmatrix \ref{tab:matrix_databases} wordt duidelijk dat de database slecht worden beoordeeld op het gebied van testbaarheid en kennis. Er is hiervoor niet genoeg ondersteuning binnen de organisatie om gebruik te maken van een database oplossing. De grootste zorg is de kans dat toenemende complexiteit van SQL queries en het gebrek aan expert kennis voor problemen zorgt.

\begin{table}[bh]
\caption{Vergelijkingsmatrix waarin eisen worden getoetst tegenover de gevonden databases}
\label{tab:matrix_databases}
\begin{tabular}{|p{3cm}|l|l|l|l|l|l|}
\hline
           & support & scalability               & productief & testbaar & kennis &     (score)  \\ \hline
GreenplumDB & +       & +                         & -          & -        & -      & 5     \\ \hline
Vertica     & +       & +                         & +          & -        & -      & 7     \\ \hline
Redshift    & +       & +                         & +          & -        & -      & 7     \\ \hline
\end{tabular}
\end{table}

\subsubsection{\textbf{Distributed oplossingen}}

In vergelijkingsmatrix \ref{tab:matrix_distributed} wordt duidelijk dat de distributed oplossingen slecht worden beoordeeld op het gebied productiviteit. De voornaamste verklaring hiervoor is ook naar voren gekomen in het onderzoek \ref{sec:distributed}. De tijd die een MapReduce job in beslag neemt is onvoorspelbaar bij adhoc queries. Dit kan betekenen dat het lang duurt voordat een developer feedback krijgt tijdens het ontwikkelen. In tegenstelling is dit geen probleem bij Spark omdat een cluster in standalone-mode op een locale laptop kan worden gestart. Daarom scoort Spark veel hoger. Over het testen van implementaties in Spark valt nog te onderzoeken. Daarnaast is het een afweging waard om code te implementeren in Python, maar uit onderzoek blijkt een implementatie in Spark SQL 10x sneller te zijn door een optimale vertaling van SQL \parencite{armbrust2015spark}.

\begin{table}[bh]
\caption{Vergelijkingsmatrix waarin eisen worden getoetst tegenover de gevonden distributed systemen}
\label{tab:matrix_distributed}
\begin{tabular}{|p{3cm}|l|l|l|l|l|l|}
\hline
                & support & scalability               & productief & testbaar & kennis &     (score)  \\ \hline
Hadoop / Hive   & +       & +                         & -          & -        & +      & 6     \\ \hline
Hadoop / Python & +       & +                         & -          & +        & +      & 7     \\ \hline
Spark SQL       & +       & +                         & +          & -        & +      & 8     \\ \hline
Spark / Python  & +       & +                        & +          & +        & +      & 8     \\ \hline
\end{tabular}
\end{table}

\clearpage

\subsubsection{\textbf{Hardware oplossingen}}

In vergelijkingsmatrix \ref{tab:matrix_hardware} wordt duidelijk dat deze programmeer talen indicatoren hebben om hoge performance te bereiken. Tijdens het onderzoek in \ref{sec:hardware} zijn er veel expert talen besproken. In vergelijkingsmatrix zijn slechts een aantal van deze talen opgenomen ter referentie, namelijk Verilog en C++ en zijn al eerder besproken als minder productief. Het onderwerp scalability moet anders worden geinterperteerd in dit geval, omdat er geen sprake is van een dataprocessing tool. Toch is de consensus dat deze talen de mogelijkheid bieden om hoge performance te behalen. Met behulp van een up-scaling methode of out-scaling door het inzetten van de aanwezige architectuur met Resqueue. Op deze manier kun data bijvoorbeeld per webshop worden verwerkt over verschillende machines, vergelijkbaar met MapReduce. De testbaarheid van programmeer talen is erg goed beoordeeld. Als laatste zijn de onderzochte DSL talen OptiSQL en Forge slecht beoordeeld door het gebrek aan beschikbare informatie, kennis en documentatie.

\begin{table}[bh]
\caption{Vergelijkingsmatrix waarin eisen worden getoetst tegenover de gevonden programmeer talen}
\label{tab:matrix_hardware}
\begin{tabular}{|p{3cm}|l|l|l|l|l|l|}
\hline
           & support & scalability               & productief & testbaar & kennis &     (score)  \\ \hline
Verilog    & -       & +                         & -          & -        & -      & 2     \\ \hline
C++        & +       & +                         & -          & -        & +      & 6     \\ \hline
OptiSQL    & -       & +                         & +          & +        & -      & 5     \\ \hline
Forge      & -       & +                         & +          & +        & -      & 5     \\ \hline
Golang     & +       & +                         & +          & +        & +      & 8     \\ \hline
\end{tabular}
\end{table}

\subsubsection{\textbf{Conclusie}}

Alle omschreven gevonden technologieën in Deelvraag 2 zijn beoordeeld. Hiermee wordt de volgende vraag beantwoord: ``Wat zijn toonaangevende methodes en technologieën om statistieken te berekenen, die passen bij de wensen en eisen van de opdracht?". Voor de beoordeling is onderzoek gedaan te lezen in hoofdstuk \ref{sec:deelvraag2}. Vervolgens zijn alle omschreven technologieën beoordeeld door middel van de checklist (zie tabel \ref{tab:checklist}) en aan de hand van groupsdiscussies.

Hieruit is gekomen dat Golang en Spark met een 8 zijn beoordeeld, daarna de databases Vertica en Redshift met een 7. Omdat er niet genoeg consensus is over het toepassen van een database oplossingen wordt deze niet getest. Dit betekend dat Golang en Spark worden getest in een tweetal Proof of Concept 's.
