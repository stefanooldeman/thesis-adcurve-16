\chapter{Planning} % de projectactiviteiten met een beschrijving van de onderlinge samenhang, de mijlpalen en een fasering in de tijd met een schatting van de te besteden uren voor de verschillende uit te voeren activiteiten;

In de planning wordt omschreven hoe de besluitvorming zal verlopen. Er is gekozen om met Agile development methodieken te werken. Daarnaast wordt verder toegelicht hoe de opdracht verder is afgebakend, zodat de student en opdrachtgever alle verwachtingen helder hebben. 

\section{De opdracht} % de afstudeeropdracht met mogelijke deelopdrachten, met vermelding van de projectgrenzen en randvoorwaarden en, indien het afstudeerproject onderdeel uitmaakt van een groter project, de afbakening t.o.v. het grotere project;

Shop2market heeft altijd voor een uitdaging gestaan als het aan komt op het verwerken van data. Er zijn veel technologieën beschikbaar, maar niets is perfect. De grootte variatie aan use-cases maakt het kiezen van een oplossing geen voor de hand liggende keuze.
Daarom is de opdracht gegeven om een Proof of concept te ontwikkelen om de risico 's van een investering te minimaliseren.


\section{Activiteiten}
\label{sec:activiteiten}
De project activiteiten zijn omschreven om de onderlinge samenhang en fasering te visualiseren. De fases hebben dezelfde indeling als de deelvragen uit het onderzoeksplan, zie sectie \ref{sec:onderzoekmethode} op pagina \pageref{sec:onderzoekmethode}. Per activiteit is een inschatting gegeven in uren. Het totaal aantal uren komt uit op 240 werkuren. Deze planning is uitgebreid terug te vinden in apendix \ref{app:planning}: Overzicht planning en fases.


\begin{enumerate}
    \item Afleggen interviews rondom Requirements \& Constraints, documenteren use-cases  (\textit{4 uur})
    \item Onderzoeken van toonaangevende, passende technologieën (\textit{24 uur})
    \item Impact analyse uitvoeren rondom use-cases (\textit{4 uur})
    \begin{enumerate}
        \item Onderzoek technische factoren die voor problemen zorgen (\textit{12 uur})
        \item Oplossing ontwerpen voor use-cases en Proof of concept (\textit{16 uur})
    \end{enumerate}
    \item Bespreken van ontwerpen en evt. aanpassen van ontwerpen (\textit{2 uur})
    \begin{enumerate}
        \item Vastleggen kwaliteitscriteria Proof of concept's (\textit{2 uur})
    \end{enumerate}
    \item Uitvoeren van Proof of concept's
    \begin{enumerate}
        \item Eerste iteratie (\textit{64 uur})
        \item Tweede iteratie (\textit{56 uur})
        \item Derde iteratie (\textit{56 uur})
    \end{enumerate}
\end{enumerate}


De ingeschatte uren zijn uitsluitend de benodigde activiteiten voor het schrijven van de scriptie. Hiervoor wordt een inschatting gedaan van 200 uur. Dit geeft 40 uur voor het bijwerken van de scriptie bij ieder van de vijf fasen. Het aantal benodigde uren voor dit project komt hierdoor uit op 440 uur.

\clearpage

\section{Resultaten}

De volgende producten dienen in ieder geval te worden opgeleverd, deze worden omschreven met de kwaliteitscriteria die bekend zijn op dit moment. In de ontwerpfase van het project worden er kwaliteitscriteria gedocumenteerd per Proof of concept. Dit wordt voor iedere nieuwe iteratie gedaan, zie apendix \ref{app:planning}.

\begin{enumerate}
\item Impact Analyses van huidige probleem scenario's \\
      \textit{Criteria: Tabel en/of visualisatie van scenario's, \\de waarschijnlijkheid en de technische en organisitorische impact hiervan }
\item Lijst van (non-)functional requirements \\
      \textit{Criteria: Geprioritiseerd volgens MoSCoW methode}
\item Planning en Ontwerp per Proof of concept met kwaliteitscriteria \\
      \textit{Criteria: Het ontwerp is besproken en goedgekeurd door de interne stakeholders.\\ De iteratie is omschreven en ingevoerd in het ticketsysteem}
\item Programmatuur van Proof of concept's \\
        \textit{Criteria: indien geslaagd, voldoet aan criteria van het ontwerp.\\ De programmatuur is altijd vastgelegd in versiebeheer}
\item Test resultaten en bevindingen uit Proof Of concept's \\
        \textit{Criteria: de vragen in het plan per Proof of concept zijn beantwoord}
\item De thesis met onderzoeksresultaten \\
    \textit{Criteria: voldoet aan de eisen gesteld in de afstudeerleidraad, \\
    relavant aan de afstudeerperiode aan de Hogeschool Utrecht: HBO-ICT}
\end{enumerate}

    \subsection{Oplevering}
    
    Voor de organisatie is de uitvoer van Proof of concept's en het resultaat hiervan het eindproduct. Hiervoor zijn meerdere opleveringen gepland gedurende het project. Intern zijn deze ingedeeld in Agile iteraties van twee weken. Zie apendix \ref{app:planning} waarin alle interne oplever data staan vernoemd onder \textit{demo}. \\
    
    Voor de opleiding is het belangrijkste eindproduct de thesis, met de nadruk op het onderzoek. Deze wordt ingeleverd bij de Hogeschool Utrecht voor 31 mei 2016. Daarnaast worden er twee concept versies gecommuniceerd, gepland op 26 april 2016 en 10 mei 2016.
    
    Concreet worden hierbij de volgende beroepsproducten aan het eind van het project opgeleverd:
    
    \begin{itemize}
    \item De thesis met alle onderzoeksresultaten.
    \item Resultaten van Proof of concept's, aldus de programmatuur.
    \item De bevindingen, te lezen in "Test resultaten en bevindingen uit Proof of concept's", zijn bij de thesis inbegrepen.
    \end{itemize}
    
\newpage

\section{Scope}

De statistieken worden momenteel berekend op een aantal dimensies: per jaar, per maand, per week en per dag. De aggregaties worden in een aantal stappen uitgevoerd, in de volgende volgorde:
\begin{itemize}
\item per geadverteerd product bij een publisher.
\item totalen per publisher
\item totalen voor de gehele webwinkel
\end{itemize}
Het berekenen van statistieken op het laagste niveau geeft de grootste uitdaging. Dit komt mede omdat de invoer data (input) een groter volume heeft op dit niveau, in vergelijking tot latere aggregaties.

Ieder Proof of concept beperkt zich daarom tot het berekenen van een aantal metrieken op het laagste aggregatie niveau voor webwinkels: geadverteerde producten.


\section{Randvoorwaarden}

Voor dit project zijn er geen afhankelijkheden van derden partijen of andere projecten binnen in de organisatie. Daarom is de student slechts afhankelijk van de samenwerking van de organisatie. Omdat alle betrokkenen al langere tijd samenwerken, kunnen de volgende voorwaarden als vanzelfsprekend gezien worden:

\begin{itemize}
    \item De organisatie houd rekening met het project en de student in de reguliere planning van iteraties. De plannig zoals in apendix  \ref{app:planning} is hiervoor het uitgangspunt.
    \item De stakeholders stellen tijd beschikbaar voor het afleggen van interviews en goedkeuring van documentatie.
    \item De student heeft toegang tot benodigde servers, data en eventuele lab omgeving.
\end{itemize}

\newpage

\section{Risico's} % de eventuele risico’s met maatregelen;

Gedurende het project zijn er een aantal momenten waarop de risico’s worden geëvalueerd. De mogelijkheid bestaat om dit dagelijks te doen tijdens de \textit{daily stand-up}. Zo kan er bijvoorbeeld tijdens de uitvoer van een Proof of concept worden gezien of de ingeslagen weg vitaal is.
Hieronder worden een aantal mogelijke scenario's uitgeschreven en de mogelijke besluitvorming.

\begin{table}[h!]
\centering
\caption{Risico scenario's met mogelijke besluitvorming om de impact te minimaliseren}
\label{tab:risks}
\def\arraystretch{1.5}
\begin{tabular}{|p{5cm}|l|p{8cm}|}
\hline
Scenario & Impact & Mogelijke onderneming \\ \hline
    Tijdens een geplande iteratie / Proof of concept wordt duidelijk dat een ontwerp niet voldoet
    & Planning 
    & Het huidige plan wordt verworpen. Er moet een nieuw ontwerp worden gemaakt. In de huidige iteratie worden nieuwe taken omschreven, ingeschat en uitgevoerd in ruil voor het vorige plan. \\
\hline
    Een van de belangrijkste use-cases in het Proof of concept zijn naar tevredenheid opgelost, met uitzondering van enkele use-cases. Daarnaast is geconcludeerd dat niet alle use-cases door een techniek kan worden opgelost.
    & Kwaliteit
    & Indien er voldoende tijd over is zal er een mogelijk een iteratie worden gewijzigd of gepland. Hierin moet een oplossing worden ontworpen en ontwikkeld om de resterende use-cases op te lossen. Indien hier geen tijd voor is, is dit scenario een waardevolle uitkomst voor de opdrachtgever. \\
\hline
    De schaalbaarheid van technologieën moet mogelijk worden bewezen, dit kan veel tijd in beslag nemen door het verplaatsen van data. Het uitvoeren van deze test moet geautomatiseerd kunnen.
    & Planning
    & Voor het geautomatiseerd testen is extra tijd nodig in de maand Mei, deze is momenteel niet gepland omdat het niet zeker is of deze activiteit vereist is. Daarnaast is het alleen van toepassing op geslaagde ontwerpen die in het Proof of concept naar voren zijn gekomen. \\
\hline
\end{tabular}
\end{table}

\clearpage

\section{De projectorganisatie} % de projectorganisatie;

De software ontwikkeling valt onder leiding van M. Jorissen. In nauwe samenwerking met M. Gill zijn zij verantwoordelijk voor de project analyse en planning. Het development team bestaat uit zes ontwikkelaars waaronder één ontwerper D. Wilpshaar.
De architectuur wordt geleid door S. Pogrebnyak. Dit team is verantwoordelijk voor het uitwerken en oplossen van projecten. Hierbij wordt gebruik gemaakt van een aantal methodieken en gewoontes vanuit Agile-softwareontwikkeling, waaronder: Kanban, Daily standup, Retrospectives, Extreme Programming (XP), Test Driven Development (TDD) en Continuous integration (CI).

    \subsection{Werkwijze en methodieken}

    Ondanks dat er in de werkwijze vaak een uitgebreide planning of requirements analyse ontbreekt, wordt de productiviteit en software kwaliteit toch bewaakt door de principes uit XP zoals TDD. Refactoring\footnote{“Code refactoring is het proces waarbij de structuur van bestaande code wordt gewijzigd zonder de functionaliteit te wijzigen”…“dit komt ten goede aan de onderhoudbaarheid van code, en creëert nieuwe architectuur of object modellen om code makkelijker te kunnen uitbreiden met nieuwe functionaliteiten“ \parencite{refactorruby}} speelt hier een grootte rol in.
    Hierdoor kan het team zich veroorloven om bij het ontwerpen en ontwikkelen van projecten minder vooruit te plannen. Software wordt hierdoor gevormd door requirements die in de huidige situatie aanwezig zijn. In tegenstelling tot het ontwikkelen met abstracties; zodat software flexibel genoeg is voor latere aanpassingen en requirements. Veel planning is hierdoor niet aanwezig en technische implementatie komt vooral tot stand door TDD en brainstorm sessies. Dit is mede mogelijk doordat de omgeving de mogelijkheid bied om fouten te maken.

    \subsection{Betrokkenen}
    
    Betreft de stakeholders van het project zijn de volgende personen betrokken bij het project: Matthijs Jorissen (Opdrachtgever) en  S. Pogrebnyak (technische eindverantwoordelijk). Daarnaast is Marco Dumont betrokken in zijn rol als docent-begeleider. S. Oldeman, de student heeft in dit project een dubbel rol als student en werknemer.

    \subsection{Werkwijze en verantwoordelijkheid student}

    Het plannen en ontwikkelen van het software project valt onder de volledige verantwoordelijkheid van de student. De planning, doelstellingen en requirements worden in overleg met M. Jorissen vastgelegd. In verband met overdracht en kennis-deling zal er afstemming plaats vinden met S. Pogrebnyak over technische afwegingen. Dit gaat door middel van code reviews, discussies en document-deling.

\section{Ethische afweging} % de projectorganisatie;

Er zijn kansen aanwezig om aspecten van het project ethische te evalueren. Dit moet echter nog verder worden onderzocht en is afhankelijk van het Proof of concept. Bijvoorbeeld doordat er gebruik wordt gemaakt van persoonlijke gegevens. Een andere mogelijkheid is dat het project op zichzelf wordt overwogen vanuit een ethisch perspectief. Bijvoorbeeld door te kijken of er ethisch is gehandeld tijdens het uitvoeren van het project. Wanneer er geen spraken is van een ethisch dillema zal dit worden toegelicht.