\chapter{Planning} % de projectactiviteiten met een beschrijving van de onderlinge samenhang, de mijlpalen en een fasering in de tijd met een schatting van de te besteden uren voor de verschillende uit te voeren activiteiten;

\section{De opdracht} % de afstudeeropdracht met mogelijke deelopdrachten, met vermelding van de projectgrenzen en randvoorwaarden en, indien het afstudeerproject onderdeel uitmaakt van een groter project, de afbakening t.o.v. het grotere project;

\section{De projectorganisatie} % de projectorganisatie;

    \subsection{Organisatie structuur}

    De software ontwikkeling valt onder leiding van M. Jorissen. In nauwe samenwerking met M. Gill zijn zij verantwoordelijk voor de project analyse en planning. Het development team bestaat uit zes ontwikkelaars waaronder één ontwerper D. Wilpshaar.
    De architectuur wordt geleid door S. Pogrebnyak. Dit team is verantwoordelijk voor het uitwerken en oplossen van projecten. Hierbij word gebruik gemaakt van een aantal methodieken en gewoontes vanuit Agile-softwareontwikkeling, waaronder: Kanban, Daily standup, Retrospectives, Extreme Programming (XP), Test Driven Development (TDD) en Continuous integration (CI).

    \subsection{Werkwijze en methodieken}

    Ondanks dat er in de werkwijze vaak een uitgebreide planning of requirements analyse ontbreekt, word de productiviteit en software kwaliteit toch bewaakt door de principes uit XP zoals TDD. Refactoring speelt hier een grootte rol in.\parencite{refactoring-ruby}
    \footnote{“Code refactoring is het proces waarbij de structuur van bestaande code word gewijzigd zonder de functionaliteit te wijzigen”…“dit komt ten goede aan de onderhoudbaarheid van code, en creëert nieuwe architectuur of object modellen om code makkelijker te kunnen uitbreiden met nieuwe functionaliteiten“}.
    Hierdoor kan het team zich veroorloven om bij het ontwerpen en ontwikkelen van projecten minder vooruit te plannen. Software word hierdoor gevormd door de requirements voor die in huidige situatie aanwezig zijn. In tegenstelling tot het ontwikkelen met abstracties zodat software flexibel genoeg is voor latere aanpassingen en requirements. Veel planning is hierdoor niet aanwezig en technische implementatie komt vooral tot stand door TDD, brainstorm sessies. Dit is mede mogelijk doordat de omgeving de mogelijkheid bied om fouten te maken.

    \subsection{Project verantwoordlijkheden}

    Het plannen en ontwikkelen van het software project omschreven in dit documet valt onder de volledige verantwoordelijkheid van de student. De doelstellingen en requirements worden in overleg met M. Jorissen vastgelegd. In verband met overdracht en kennis deling zal er afstemming plaats vinden met S. Pogrebnyak technische aspecten. Bijvoorbeeld over de aansluiting op huidige visie en de productie omgeving en mogelijke aanpassingen of toevoegingen aan de infrastructuur.
    De student ziet het als zijn verantwoordelijkheid is om op innovatieve wijze software te ontwikkelen.


\section{Activiteiten} % de projectactiviteiten met een beschrijving van de onderlinge samenhang, de mijlpalen en een fasering in de
tijd met een schatting van de te besteden uren voor de verschillende uit te voeren activiteiten;

\section{Risico's} % de eventuele risico’s met maatregelen;

\section{Oplevering}
% twee data voor het inleveren van concepten van de scriptie (circa 6 resp. 3 weken voor de inleverdatum van

de definitieve versie van de scriptie);
