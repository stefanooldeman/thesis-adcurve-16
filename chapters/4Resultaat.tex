\chapter{Resultaat}
\label{ch:resultaat}

Eerder is er in \ref{sec:gevonden_tools} geconcludeerd dat dit met behulp van Golang en Spark wordt ontwikkeld.

In dit hoofdstuk worden eerst de probleem scenario 's behandeld, en wordt afgeleid wat de kernoorzaken hiervoor zijn. Door deze analyse is gevalideerd hoe een implementatie in Golang of Spark deze scenario 's kan voorkomen, tevens wordt er een advies gegeven over mogelijke vervolg stappen die niet in dit project zijn verwerkt. Dit geeft antwoord op deelvraag 3 en is te lezen in \ref{sec:deelvraag3}.

Het tweede deel is te lezen in \ref{sec:deelvraag4}, hier wordt omschreven hoe de Proof of Concept ’s zijn uitgevoerd.

Na een laatste toets op volledigheid wordt een oplossing gekozen en verantwoord. Dit geeft antwoord op \ref{sec:deelvraag5}.
\clearpage

% ACTIVITY Impact analyse uitvoeren rondom use-cases
\section{Problematische scenario 's en mogelijke oplossingen}
\label{sec:deelvraag3}
\textit{Wat zijn de probleem scenario's waardoor gegevens niet te verklaren zijn, en wat zijn de mogelijke oplossingen?}

Tijdens de project analyse is met verschillende stakeholders binnen de organisatie gesproken. Naast de gedefinieerde projecteisen zijn er verschillende symptomen verzameld. Deze zijn gebruikt voor het samenstellen van een Impact analyse gemaakt, zie Apendix \ref{app:impact_analyse}

Vervolgens is per probleemscenario een mogelijke technische verklaring gezocht, zijn er criteria omschreven waaraan de gebruikte technologie bij een mogelijk oplossing moet voldoen. Hierdoor wordt getoetst of de geselecteerde technologieën in \ref{sec:gevonden_tools} toepasbaar zijn en de huidige probleem scenario werkelijk kunnen oplossen.

\subsection{Scenario's uit de Impact Analyse}

\begin{enumerate}
    \item L6: API 's provided by the publisher are unavailable
    \item M5: API 's for google, change frequently and the import stops working, or fail silent (values are not matched,,resulting in empty fields)
    \item H6: Wrong costs by configuration, issues: mistakes in tracking Order amount, tracked wrong, multiplied by 100 bug
    \item L7: Metrics calculated incorrectly because of mismatching product data (ids from tracking don't match cost exports from publisher)
    \item L7: Calculations fail,because technology or infrastructure 
    \item L4: Aggregations from different dimensions on the same data
    don't match
    \item H7 Bij het bekijken van dashboards in Adcurve zijn de gegevens over de vorige dag pas beschikbaar na 12:30 CET. Omdat alle data in een keer wordt verwerkt wordt het proces gestart nadat alle afhankelijke data bronnen beschikbaar zijn.
\end{enumerate}

Scenario een (1) tot en met vier (4) zijn niet te voorkomen omdat de organisatie zelf hier weinig invloed op kan hebben. Het wordt veroorzaakt door externe processen bij publishers, of mensen in diesnt van de webwinkel die de advertenties of webwinkel integraties verkeerd configureren. Scenario vijf (5) en zes (6) hebben een hoge impact en zijn specifieke problemen met de huidige implementatie. Deze scenario's zijn te voorkomen door het inzetten van andere technologie of een andere strategie als het aan komt op data processing.
Scenario 7 is eerder besproken tijdens de non functionele eisen in \ref{table:requirements}. Deze is opgenomen zodat onderzocht wordt hoe dit het best kan worden voorkomen.  

\clearpage

\subsection{Technische te verklaren oorzaken}

Met betrekking tot scenario vijf (5) en zeven (6) zijn verschillende problemen gevonden in relatie tot de gebruikte versie van MongoDB \parencite{mongo_changelog}. Uit monitoring systemen zijn de foutmeldingen terug te relateren aan software bugs die aanwezige zijn in versie 2.4.

De consensus binnen de organisatie is dat een upgrade naar een volgende versie riskant is. Het is niet in te schatten hoe veel problemen er zullen optreden. Daarnaast moeten er processen worden ingericht om het synchroniseren van order te ondersteunen en data verlies te voorkomen wanneer de database offline moet.
% TODO: is het mogelijke om machines van verschillende versies naast elkaar te draaien in een cluster?
    
Met betrekking tot scenario zeven (7) zijn de volgende mogelijke oorzaken gevonden:

\begin{itemize}
    \item Uit monitoring systemen blijkt dat het online productie processen van invloed zijn op de performance van het aggregatie proces. Omdat MongoDB direct wordt gebruikt voor het synchroniseren van bestellingen met webwinkels, en dit een realtime proces is, wordt het cluster hiermee belast. In de huidige situatie worden de MapReduce jobs op een aantal machines in het cluster uitgevoerd, toch word de performance beïnvloed.
    \item het aggregeren van data kan in MongoDB wellicht trager zijn dan nodig omdat de tussentijdse resultaten uit de fases in MapReduce worden weggeschreven naar verschillende machines \parencite{mongo_mr_shards}. Zo valt ook af te lezen uit monitoring dat het wegschrijven van de uiteindelijk resultaten langer duurt dan het aggregeren zelf. Dit heeft mogelijk te maken met het bijhouden van indexes.  \parencite{mongo_write_performance}
    % replica sets? sharding? grootte cluster?
\end{itemize}


\subsubsection{\textbf{Conclusie}}
\label{subsec:3.3.2}

Tijdens de root cause analysis zijn een aantal oorzaken omschreven. Hieruit valt te concluderen dat een mogelijke nieuwe oplossing aan de volgende criteria moet voldoen:

\begin{enumerate}[label=(\alph*)]

    % Er rekening moet worden gehouden met de mogelijke overhead van networking applications, en tegelijkertijd de mogelijkheid moet zijn om grootte data sets op te splitsen zodat er geen ``disk spills"\ optreden.
    \item In de nieuwe situatie moet op efficiënte wijze worden om gegaan met een kleine hoeveelheid aan data, zodat de verschillende fases van het aggregeren (zoals map, reduce, shuffle en sort). Dit omdat voor 80\% van de webshops een kleine hoeveelheid data wordt verzameld. Tegelijkertijd is de input data set groot en zijn resultaten van aggregaties wellicht groter. Er moet in deze situatie op efficiënte wijze data worden weg geschreven naar een eindlocatie,  zoals het filesysteem. Hierbij moet het onderhouden van indexes worden voorkomen.
    
    \item Er moet situaties situatie worden voorkomen waarin het onmogelijk is om een technologie of systeem te upgraden naar een nieuwere versie. Dit is te voorkomen door a) geen state te bewaren in het systeem zelf, maar dit alleen te gebruiken bij data processing. b) door het systeem onafhankelijk te maken van productie / online processen. Dit biedt de mogelijkheid door bijvoorbeeld een database back-up te doen, het systeem te migreren en de data opnieuw in te laden.
\end{enumerate}

\clearpage


\subsection{Hoe wordt dit voorkomen met de gevonden tools / POC?}
\label{subsec:deelvraag3_vergelijking}


\subsubsection{\textbf{Versie upgrades}}

Door de tools binnen het Hadoop ecosysteem wordt geen state bewaard. Maar alle data wordt beheerd door HDFS en opgeslagen als toegankelijke bestanden in een gekozen bestandsformaat. Het risico is dat het upgraden van versies, niet compatible met verschillende bestands formaten. Ook de tools die opereren op het data formaat moeten compatible zij. Daarnaast bestaat de mogelijkheid dat HDFS data anders beheerd onder verschillende versies. Hierdoor zijn data migraties erg operationeel intensief is. Hierdoor is het sterk aan te raden om dit niet zelf te beheren, maar gebruik te maken van distributie zoals Amazon EMR, Hortonworks, Cloudera of MapR.

Omdat Spark onder andere uitmaakt van het Hadoop ecosysteem zijn deze conclusies te generaliseren.

Omdat databases bij definitie statefull zijn, moet er in dit geval een back-up en restore mogelijkheid aanwezig zijn om upgrades te kunnen uitvoeren.

In het geval van Golang, dat in het algemeen niet als data processing tool te beschouwen is maar als een General purpose language, is de implementatie code eenvoudig te upgraden naar nieuwere versies. Dit wordt gegarandeerd als de implementatie volledig getest is door unit tests. 


\subsubsection{\textbf{Mogelijke overhead voorkomen}}

\begin{verbatim}

Use the COST PAPER (single threading vs. network overhead)

https://www.usenix.org/conference/hotos15/workshop-program/presentation/mcsherry

AND The paper measuring NETWORK OPTIMIZATIONS vs CPU/MEM in DATA PROCESSING TOOLS VS CPU
https://www.usenix.org/system/files/conference/nsdi15/nsdi15-paper-ousterhout.pdf

Reference to the paper and explain:
SPARK USES NO INDEX \parencite{armbrust2015spark}
MAP REDUCE (Google / Apache) USES NO INDEX 
SHARED NOTHING DATABASES USE NO INDEXES \parencite{lamb2012vertica} of \parencite{harizopoulos2008oltp} / \parencite{dewitt2006build}

MAYBE ALSO APPLY 'Amdahls law'
https://en.wikipedia.org/wiki/Amdahl%27s_law

 AND MAYBE 
"roughly 80\% of the effects come from 20\%"
https://en.wikipedia.org/wiki/Pareto_principle

\end{verbatim}


\subsubsection{\textbf{Oplosing om  high query speed en good analytics performance te behalen met  twee losse systemen}}

Het splitsen van data verwerken en data opslag / opvragen betekend dat de geaggregeerde data zal worden opgeslagen in s3, zodat de bestaande API de data kan uitserveren. 
De aggregaties op basis van week, jaar en maand komen te vervallen. De overige aggregatie die nodig is om API client snel te houden wordt gedaan door Saki. Doordat de bestanden gesorteerd worden opgeslagen op s3, kan saki deze binnen 100 milliseconden aggregeren per query.

Dit heeft als gevolg dat data opslag volledig verplaatst naar Amazon S3, zowel de input data als de output data. (data pipeline)

    
\subsection{Conclusie}

Een deel van het probleem is het actief monitoren van de data kwaliteit en processen starten om het te corrigeren. Hier wordt niet verder op in gegaan omdat een dergelijke oplossing hiervoor buiten scope valt. Hiervoor geldt, zodra er een oplossing is moeten de datakwaliteit worden herstelt door het proces te starten dat de correcties uitvoert [``effected datasources"\ opnieuw evalueren]
Deze scenario's zijn ook tegelijkertijd lastig te voorkomen, alleen te herstellen. Het advies is daarom om hier ``monitoring en alerting\ voor te implementeren en verder onderzoek doen naar bestaande methodieken voor beheersen van datakwaliteit. Bijvoorbeeld door starten van een ``Data Stewardship Program" zoals omschreven wordt in het ``TDWI Report, The Data Warehouse Institute"\ door  \textcite{eckerson2002data}.

Spark SQL is bewezen snel erg snel te zijn en Amazon EMR is managed service

database column store heeft geen indexes maar partities, bewezen erg snel te zijn… maar geen voordelen worden behaald omdat  ons proces altijd N-max aantal columns worden opgevraagd. Daarom zal de query nooit sneller zijn, unless er meerdere query 's zijn waarbij verschillende columnen worden opgevraagd. Daarom zijn er te weinig voordelen om hier een POC mee te starten, daarnaast zullen versie upgrades lastig zijn. 

De Aggregatie processen met Golang zullen erg klein worden gemaakt door de data van te voren te partitioneren per shop. Dit biedt de mogelijkheid om het deze processen uit te voeren op verschillende machines (scale-out), door het gebruikt van resqueue workers "Resque workers can be distributed between multiple machines, support priorities, are resilient to memory bloat / "leaks," \parencite{github2016reque} 


Kort: Er wordt niet verder geïnvesteerd in een POC met databases omdat er te veel keuze is, en technologie niet effectief is toe te passen

Wel wordt er een POC uitgevoerd met Golang en Spark


% Het splitsen van data verwerken en een databases waaruit de data is op te vragen, heeft met gevolg dat Saki, een applicatie die in real time aggregaties op het hoogste niveau pakt (channel, shop, shop_product_id) en deze flexibel aggregeerd op basis van datum.
\clearpage

% ACTIVITY Bespreken van ontwerpen en evt. aanpassen van ontwerpen
% ACTIVITY Vastleggen kwaliteitscriteria Proof of concept's
\section{Ontworpen experimenten middels proof of concept}
\label{sec:deelvraag4}
% \textbf{Hardware oplossing met Golang}
% Het is praktisch mogelijk om data in sub sets te aggregeren omdat er een afzonderlijke data bron is per publisher.
% Het gebruiken van gesplitste bestanden per webshop zorgt voor zeer korte levende processen. De totale performance hoeft niet te verbeteren, zolang het proces dat alleen data aggregatie uitvoert voor de specifieke webwinkel en publisher kort van duur is kan dit worden uitgevoerd op verschillende machines tegelijk om dit schaalbaar in te zetten. "We typically find that the highest performance is achieved when multiple threads are used per core. For Core i7, the best performance comes from running 8 threads on 4 cores" \parencite{lee2010debunking}

% \textbf{Distributed oplossing met Spark}
% Door gebruik te maken van distributed functionaliteiten van Spark wordt getest of de snelheid waar spark bekend om staat (paper) behaald kan worden in onze use case. Er is de mogelijkheid om unit tests en functional tests te schrijven, dit wordt aangeboden door het framework. Het wegschrijven van meerdere resultaten in een proces kan door het partitionener van output data op (PublisherID, ShopID). Ook kunnen er meerdere dagen in een proces worden verwerkt en het resultaat wordt weggeschreven per TimeID, PublisherID en ShopID


In dit hoofdstuk wordt omschreven hoe de Proof of concept 's zijn ontworpen. Zoals is te lezen in \ref{sec:deelvraag3} zijn de best mogelijke oplossingen te ontwikkelen met behulp van \textit{Golang} en \textit{Apache Spark}. In overeenstemming met de organisatie zijn verdere succesfactoren gedefinieerd per POC.


% De uitvoer van het project bestaat uit drie fases; Data preperation, Go Aggregate en Spark SQL
% In fase 1 vindt de ETl processing plaats. Als de resultaten snel genoeg zijn, zal dit worden hergebruikt in fase 2 en 3.
% TODO Leg uit dat Spark SQL niet wordt aangeraden voor ETL wanneer dit alleen mapping fases heeft


\subsection{Ontworpen experimenten}

Zoals wordt omschreven in de originele doelstelling in \ref{sec:doelstelling} ``moet data op tijd worden verwerkt evenals het tijdig herstellen van fouten mogelijk zijn".  Daarom valideert dit experiment welke oplossing de beste prestatie levert in de huidige situatie. Voor het meten van performance worden er drie verschillende type aggregaties uitgevoerd op "commodity hardware", de exacte specificaties worden later omschreven in sectie \ref{subsec:hardware_specs}.  Er wordt gebruik gemaakt van een productie data set van één willekeurige dag.

Tijdens iedere aggregatie worden er twee metingen uitgevoerd: de totale proces tijd \footnote{tijd word gemeten door unix `time` zie de documentatie voor time: \url{http://linux.die.net/man/1/time}} en de  tijd zonder ``warmup time".  Warmup time wordt in dit experiment omschreven als de tijd tussen het starten van het proces en het werkelijk uitvoeren van de geschreven implementatie. Voor het evalueren van de twee POC 's is besloten om geen latency, IO, memory of CPU gerelateerde benchmarks te evalueren. Dit is wel gewoonlijk bij \textbf{industrie benchmarks} van verschillende data processing tools. Maar dit wordt gedaan vanuit een competitief perspectief  \parencite{ousterhout2015making}. Voor de doelstellingen van dit project is de totale proces tijd belangrijker.

\subsection{Gebruikte data sets}

Tijdens de drie fases van het experiment wordt gebruik gemaakt van een data set uit productie als data input. De data set bestaat uit visits en orders van 264 webwinkels.

\begin{itemize}
    \item Het \textbf{orders} bestand is origineel 0.2494 GB (249.0 MB) groot, dit is in een Binary JSON (BSON) compressie formaat, Na het ETL waarbij alleen de alleen de relevantie metrieke overblijven is de dataset 0.04734 GB (5.0 MB) groot. Dit is een JSON formaat waarbij alle extra kolommen voor visits ook zijn inbegrepen en een default waarde hebben.

    \item Het \textbf{visits} bestand is origineel 1.0GB groot, dit is in een CSV bestand waarbij de TAB charachter als delimiter wordt gebruikt.  Na het ETL waarbij alleen de alleen de relevantie metrieken overblijven is de dataset 0.04734 GB groot. Dit is hetzelfde JSON formaat waarbij alle extra kolommen voor orders zijn inbegrepen en een default waarde hebben.
\end{itemize}

Het resultaat van de ETL voor orders en visits wordt als input data gebruikt voor iedere aggregatie. De drie aggregaties zijn variaties waarbij dezelfde input data wordt gebruikt, maar waarvan het resultaat verschilt per aggregatie in data grootte. Dit is om de huidige situatie en business needs te simuleren. Een grotere data set past wellicht wel of niet in het geheugen (RAM). De drie variaties maken gebruik van de volgende keys waarop de aggregatie functie \verb=SUM(metric)= wordt toegepast:

\begin{itemize}
    \item TimeId, ChannelID
    \item TimeId, ChannelID, ShopId
    \item TimeId, ChannelID, ShopId, ShopProductId
\end{itemize}

Ieder opdracht (aggregatie) wordt in één serie vijf keer herhaalt, waarvan een vervolgens de twee metingen worden geregistreerd. Over de metingen wordt een gemiddelde genomen voor het resultaat.

\subsection{Gebruikte Hardware}
\label{subsec:hardware_specs}

Tijdens het onderzoek naar verschillende technologiën zijn ook benchmarks gevonden. Hierbij wordt er vaak gepretendeert dat er een schaalbaarheid kan worden bereikt door het gebruiken van "Commodity hardware". Hiermee worden machines bedoelt die algemeen beschikbaar zijn, geen bijzondere performance voordelen bieden. Om in deze trend te blijven is er voor gekozen om geen speciale omgeving te installeren voor de experimenten. Uiteindelijk is er gebruik gemaakt van een Macbook Pro met de volgende specificaties:

\begin{table}[h]
\caption{Hardware specificaties van de gebruikte machine}
\label{tab:hardware_specs}
\begin{tabular}{ll}
Merk:      & MacBook Pro (Retina, 13-inch, Mid 2014) \\
Processor: & 2.6 GHz Intel Core i5                   \\
Geheugen:  & 8 GB 1600 MHz DDR3                      \\
Storage:   & Apple SSD                                  
\end{tabular}
\end{table}

\subsection{ETL Process, Data preparation}

Het proces dat voorziet in de eerste fase van het proces is geschreven in Golang. Hier zal voor de rest van het document naar worden gerefereerd als de splitter. ETL (Extract, Transform and Load) is een manier om data uit een externe databron te transformeren naar een ander schema   \parencite{data-mining}. Dit zorgt voor een uniform of formaat waardoor latere processen geen logica hoeven toe te passen om met de data te werken. In deze context worden er alleen met bestanden gewerkt, de dataflow gaat als volgt:

\begin{enumerate}
    \item Extract / download bestanden van Amazon S3
    \item Transform / bewerk de bestanden op locale hardeschijf met de splitter implementatie in Golang waardoor een BSON of CSV formaat wordt omgezet naar JSON met een uniform schema.
    \item Load / het resultaat wordt weg geschreven naar de locale hardeschijf. In het ontwerp is er voor gekozen om data te partitioneren per \verb+shop_id+ voor later gebruik.
\end{enumerate}

Ter illustratie is in tabel \ref{tab:etl_input_example} worden een aantal kolommen uit de input data gebruikt, in dit geval uit de CSV met visits. Iedere regel in dit bestand representeert een bezoek op de website van een webshop. Deze bezoeken komen voort uit advertenties die door Adcurve zijn gepubliceerd bij de publishers.

\begin{table}[bh]
\centering
\caption{Een voorbeeld van de input data gebruikt tijdens ETL}
\label{tab:etl_input_example}
\begin{tabular}{|l|l|l|l|l|l|}
\hline
timestamp  & time\_id & publisher\_id & shop\_id & shop\_product\_id & shop\_category\_id \\ \hline
1464154281 & 20160424 & 410         & 224      & 103774145         & 338790             \\ \hline
1464154306 & 20160424 & 61          & 910      & 108837485         & 6782117            \\ \hline
1464154314 & 20160424 & 410         & 224      & 100670758         & 9152995            \\ \hline
\end{tabular}
\end{table}

Zoals te zien is in figuur \ref{fig:visits.json_after_etl} wordt iedere regel uit het CSV bestand getransformeerd naar het JSON formaat. Daarnaast wordt er een extra kolom toegevoegd: \verb+Traffic=1+, naast de andere kollomen die dienen als ``placeholders"\ voor de orders data. Dit is een bewuste keuze in het ontwerp voor het behalen van mogelijke betere performance. Dit komt ook ten goede van de data consistentie omdat de data interpretatie van data in één fase plaatsvindt. De vervolg fases kunnen hierdoor gebruik maken van pure functies die in een framework, technologie of taal zijn geïmplementeerd.

\begin{figure}[htb]
% \centering
\caption{De bestanden en hun formaat na het ETL proces door de Splitter}
\label{fig:visits.json_after_etl}
\begin{verbatim}
/Users/stefano/dev/go-aggregate/exports/224/visits.json
{"TimeId":"20160424","OrderID":"","ShopID":224,"ChannelID":410,
"ShopProductID":103774145,"ShopCategoryID":338790,"Traffic":1,
 "Order":0,"Amount":0,"ExAmount":0,"Cost":0}\n
{"TimeId":"20160424","OrderID":"","ShopID":224,"ChannelID":410,
 "ShopProductID":100670758,"ShopCategoryID":9152995,"Traffic":1,
 "Order":0,"Amount":0,"ExAmount":0,"Cost":0}\%
    
/Users/stefano/dev/go-aggregate/exports/910/visits.json
{"TimeId":"20160424","OrderID":"","ShopID":910,"ChannelID":61,
 "ShopProductID":108837485,"ShopCategoryID":6782117,"Traffic":1,
 "Order":0,"Amount":0,"ExAmount":0,"Cost":0}\%
\end{verbatim}
\end{figure}

Als laatste is er voor gekozen om data op te splitsen per webshop als een methode van data partitionering. Dit biedt de mogelijkheid om in latere fases gebruik te maken van concurrency of parallellisatie implementaties, omdat data kan worden verwerkt in een proces per partitie.

Als laatste biedt dit ontwerp de mogelijkheid om de data voor een specifieke webshop opnieuw te verwerken. Dit betekend dat data eenmalig wordt gefilterd. 

Door de bestanden te bewaren op de hardeschijf voor de laatste 30 dagen wordt er nog meer voordeel behaald.
In het scenario wanneer er data correcties worden uitgevoerd voor één data bron op een specifieke datum, worden er minder bestanden verwerkt tijdens de ETL omdat de andere overige data bronnen al eerder zijn verwerkt. Dit betekend dat data eenmalig wordt gefilterd.

\subsection{Golang}

De input voor dit proces is het resultaat van de ETL fase. Dit proces representeerd de reduce fase van een MapReduce algoritme. Om het proces te starten wordt de locatie van een de data partities van shop. Ter voorbeeld: \newline\verb+./bin/go_aggregator ./exports/1001/+

in bijlage \ref{app:golang_code}


- wat zijn de eisen die zijn besproken?

- de oplossing is als volgt ontworpen

- de gevonden resultaten

- warmup time is in gecompileerde Go code niet van toepassing.
Omdat Golang compileert naar assembly is er geen sprake van warmup time

\clearpage

\subsection{Spark}

- wat zijn de eisen die zijn besproken

- de oplossing is als volgt ontworpen

- de gevonden resultaten

De warmup tijd in spark is te verklaren doordat de Spark SQL syntax wordt vertaalt naar Java, het de java code wordt vertaal naar systeem code door de JVM.

\clearpage

% ACTIVITY Uitvoeren van Proof of concept's
\section{Conclusies}
\label{sec:deelvraag5}
Wat zijn de gepresenteerde oplossingen en waarom zijn deze volledig of niet?

\textbf{Go Aggregator}
Split all orders and visits for all shops (264 shops)

\begin{lstlisting}
2016/05/11 11:38:45 Begin
2016/05/11 11:40:09 Main Done
./start.sh  69.83s user 16.12s system 101% CPU 1:24.70 total

total time: 1 minute 9 seconds
\end{lstlisting}

\textbf{Spark all shop products, file per shop}

(one file contains all products for all channels)

\begin{lstlisting}
16/05/11 10:09:58 INFO SparkContext: Running Spark version 1.6.0
[2016-05-11 10:10:03] Start loading data
[2016-05-11 10:12:24] Finished writing data
Total time:     0:02:21.378717
16/05/11 10:12:24 INFO ShutdownHookManager: Deleting directory /private/var/folders/tp/….

time without job submit: 2 minutes 21 seconds
\end{lstlisting}


\textbf{Conclusie}

[todo]

Het gebruik van big data systemen is zeer schaalbaar, maar introduceerd een hoop overhead. (paper).

Voordelen van een programmeer taal is dat complexe business logic hierin kan worden geschreven zonder rekening te houden met veel abstracties zoals in parallel programming zoals map reduce of sql.. In zo'n paradigm moet een probleem naar code worden vertaalt met extra kennis naast het programmeren, kennis over het map reduce paradigm, sql join. Daarnaast wordt code vertaalt naar een query plan en het gebruik van indexes etc..


