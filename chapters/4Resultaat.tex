\chapter{Resultaat}
\label{ch:resultaat}

Eerder is er in \ref{sec:gevonden_tools} geconcludeerd dat dit met behulp van Golang en Spark wordt ontwikkeld.

In dit hoofdstuk worden eerst de probleem scenario 's behandeld, en wordt afgeleid wat de kernoorzaken hiervoor zijn. Door deze analyse is gevalideerd hoe een implementatie in Golang of Spark deze scenario 's kan voorkomen, tevens wordt er een advies gegeven over mogelijke vervolg stappen die niet in dit project zijn verwerkt. Dit geeft antwoord op deelvraag 3 en is te lezen in \ref{sec:deelvraag3}.

Het tweede deel is te lezen in \ref{sec:deelvraag4}, hier wordt omschreven hoe de Proof of Concept ’s zijn uitgevoerd.

Na een laatste toets op volledigheid wordt een oplossing gekozen en verantwoord. Dit geeft antwoord op \ref{sec:deelvraag5}.
\clearpage

% ACTIVITY Impact analyse uitvoeren rondom use-cases
\section{Problematische scenario 's en mogelijke oplossingen}
\label{sec:deelvraag3}
\textit{Wat zijn de probleem scenario's waardoor gegevens niet te verklaren zijn, en wat zijn de mogelijke oplossingen?}

Tijdens de project analyse is met verschillende stakeholders binnen de organisatie gesproken. Naast de gedefinieerde projecteisen zijn er verschillende symptomen verzameld. Deze zijn gebruikt voor het samenstellen van een Impact analyse gemaakt, zie Apendix \ref{app:impact_analyse}

Vervolgens is per probleemscenario een mogelijke technische verklaring gezocht, zijn er criteria omschreven waaraan de gebruikte technologie bij een mogelijk oplossing moet voldoen. Hierdoor wordt getoetst of de geselecteerde technologieën in \ref{sec:gevonden_tools} toepasbaar zijn en de huidige probleem scenario werkelijk kunnen oplossen.

\subsection{Scenario's uit de Impact Analyse}

\begin{enumerate}
    \item L6: API 's provided by the publisher are unavailable
    \item M5: API 's for google, change frequently and the import stops working, or fail silent (values are not matched,,resulting in empty fields)
    \item H6: Wrong costs by configuration, issues: mistakes in tracking Order amount, tracked wrong, multiplied by 100 bug
    \item L7: Metrics calculated incorrectly because of mismatching product data (ids from tracking don't match cost exports from publisher)
    \item L7: Calculations fail,because technology or infrastructure 
    \item L4: Aggregations from different dimensions on the same data
    don't match
    \item H7 Bij het bekijken van dashboards in Adcurve zijn de gegevens over de vorige dag pas beschikbaar na 12:30 CET. Omdat alle data in een keer wordt verwerkt wordt het proces gestart nadat alle afhankelijke data bronnen beschikbaar zijn.
\end{enumerate}

Scenario een (1) tot en met vier (4) zijn niet te voorkomen omdat de organisatie zelf hier weinig invloed op kan hebben. Het wordt veroorzaakt door externe processen bij publishers, of mensen in diesnt van de webwinkel die de advertenties of webwinkel integraties verkeerd configureren. Scenario vijf (5) en zes (6) hebben een hoge impact en zijn specifieke problemen met de huidige implementatie. Deze scenario's zijn te voorkomen door het inzetten van andere technologie of een andere strategie als het aan komt op data processing.
Scenario 7 is eerder besproken tijdens de non functionele eisen in \ref{table:requirements}. Deze is opgenomen zodat onderzocht wordt hoe dit het best kan worden voorkomen.  

\clearpage

\subsection{Technische te verklaren oorzaken}

Met betrekking tot scenario vijf (5) en zeven (6) zijn verschillende problemen gevonden in relatie tot de gebruikte versie van MongoDB \parencite{mongo_changelog}. Uit monitoring systemen zijn de foutmeldingen terug te relateren aan software bugs die aanwezige zijn in versie 2.4.

De consensus binnen de organisatie is dat een upgrade naar een volgende versie riskant is. Het is niet in te schatten hoe veel problemen er zullen optreden. Daarnaast moeten er processen worden ingericht om het synchroniseren van order te ondersteunen en data verlies te voorkomen wanneer de database offline moet.
% TODO: is het mogelijke om machines van verschillende versies naast elkaar te draaien in een cluster?
    
Met betrekking tot scenario zeven (7) zijn de volgende mogelijke oorzaken gevonden:

\begin{itemize}
    \item Uit monitoring systemen blijkt dat het online productie processen van invloed zijn op de performance van het aggregatie proces. Omdat MongoDB direct wordt gebruikt voor het synchroniseren van bestellingen met webwinkels, en dit een realtime proces is, wordt het cluster hiermee belast. In de huidige situatie worden de MapReduce jobs op een aantal machines in het cluster uitgevoerd, toch word de performance beïnvloed.
    \item het aggregeren van data kan in MongoDB wellicht trager zijn dan nodig omdat de tussentijdse resultaten uit de fases in MapReduce worden weggeschreven naar verschillende machines \parencite{mongo_mr_shards}. Zo valt ook af te lezen uit monitoring dat het wegschrijven van de uiteindelijk resultaten langer duurt dan het aggregeren zelf. Dit heeft mogelijk te maken met het bijhouden van indexes.  \parencite{mongo_write_performance}
    % replica sets? sharding? grootte cluster?
\end{itemize}


\subsubsection{\textbf{Conclusie}}
\label{subsec:3.3.2}

Tijdens de root cause analysis zijn een aantal oorzaken omschreven. Hieruit valt te concluderen dat een mogelijke nieuwe oplossing aan de volgende criteria moet voldoen:

\begin{enumerate}[label=(\alph*)]

    % Er rekening moet worden gehouden met de mogelijke overhead van networking applications, en tegelijkertijd de mogelijkheid moet zijn om grootte data sets op te splitsen zodat er geen ``disk spills"\ optreden.
    \item In de nieuwe situatie moet op efficiënte wijze worden om gegaan met een kleine hoeveelheid aan data, zodat de verschillende fases van het aggregeren (zoals map, reduce, shuffle en sort). Dit omdat voor 80\% van de webshops een kleine hoeveelheid data wordt verzameld. Tegelijkertijd is de input data set groot en zijn resultaten van aggregaties wellicht groter. Er moet in deze situatie op efficiënte wijze data worden weg geschreven naar een eindlocatie,  zoals het filesysteem. Hierbij moet het onderhouden van indexes worden voorkomen.
    
    \item Er moet situaties situatie worden voorkomen waarin het onmogelijk is om een technologie of systeem te upgraden naar een nieuwere versie. Dit is te voorkomen door a) geen state te bewaren in het systeem zelf, maar dit alleen te gebruiken bij data processing. b) door het systeem onafhankelijk te maken van productie / online processen. Dit biedt de mogelijkheid door bijvoorbeeld een database back-up te doen, het systeem te migreren en de data opnieuw in te laden.
\end{enumerate}

\clearpage


\subsection{Hoe wordt dit voorkomen met de gevonden tools / POC?}
\label{subsec:deelvraag3_vergelijking}


\subsubsection{\textbf{Versie upgrades}}

Door de tools binnen het Hadoop ecosysteem wordt geen state bewaard. Maar alle data wordt beheerd door HDFS en opgeslagen als toegankelijke bestanden in een gekozen bestandsformaat. Het risico is dat het upgraden van versies, niet compatible met verschillende bestands formaten. Ook de tools die opereren op het data formaat moeten compatible zij. Daarnaast bestaat de mogelijkheid dat HDFS data anders beheerd onder verschillende versies. Hierdoor zijn data migraties erg operationeel intensief is. Hierdoor is het sterk aan te raden om dit niet zelf te beheren, maar gebruik te maken van distributie zoals Amazon EMR, Hortonworks, Cloudera of MapR.

Omdat Spark onder andere uitmaakt van het Hadoop ecosysteem zijn deze conclusies te generaliseren.

Omdat databases bij definitie statefull zijn, moet er in dit geval een back-up en restore mogelijkheid aanwezig zijn om upgrades te kunnen uitvoeren.

In het geval van Golang, dat in het algemeen niet als data processing tool te beschouwen is maar als een General purpose language, is de implementatie code eenvoudig te upgraden naar nieuwere versies. Dit wordt gegarandeerd als de implementatie volledig getest is door unit tests. 


\subsubsection{\textbf{Mogelijke overhead voorkomen}}

\begin{verbatim}

Use the COST PAPER (single threading vs. network overhead)

https://www.usenix.org/conference/hotos15/workshop-program/presentation/mcsherry

AND The paper measuring NETWORK OPTIMIZATIONS vs CPU/MEM in DATA PROCESSING TOOLS VS CPU
https://www.usenix.org/system/files/conference/nsdi15/nsdi15-paper-ousterhout.pdf

Reference to the paper and explain:
SPARK USES NO INDEX \parencite{armbrust2015spark}
MAP REDUCE (Google / Apache) USES NO INDEX 
SHARED NOTHING DATABASES USE NO INDEXES \parencite{lamb2012vertica} of \parencite{harizopoulos2008oltp} / \parencite{dewitt2006build}

MAYBE ALSO APPLY 'Amdahls law'
https://en.wikipedia.org/wiki/Amdahl%27s_law

 AND MAYBE 
"roughly 80\% of the effects come from 20\%"
https://en.wikipedia.org/wiki/Pareto_principle

\end{verbatim}


\subsubsection{\textbf{Oplosing om  high query speed en good analytics performance te behalen met  twee losse systemen}}

Het splitsen van data verwerken en data opslag / opvragen betekend dat de geaggregeerde data zal worden opgeslagen in s3, zodat de bestaande API de data kan uitserveren. 
De aggregaties op basis van week, jaar en maand komen te vervallen. De overige aggregatie die nodig is om API client snel te houden wordt gedaan door Saki. Doordat de bestanden gesorteerd worden opgeslagen op s3, kan saki deze binnen 100 milliseconden aggregeren per query.

Dit heeft als gevolg dat data opslag volledig verplaatst naar Amazon S3, zowel de input data als de output data. (data pipeline)

    
\subsection{Conclusie}

Een deel van het probleem is het actief monitoren van de data kwaliteit en processen starten om het te corrigeren. Hier wordt niet verder op in gegaan omdat een dergelijke oplossing hiervoor buiten scope valt. Hiervoor geldt, zodra er een oplossing is moeten de datakwaliteit worden herstelt door het proces te starten dat de correcties uitvoert [``effected datasources"\ opnieuw evalueren]
Deze scenario's zijn ook tegelijkertijd lastig te voorkomen, alleen te herstellen. Het advies is daarom om hier ``monitoring en alerting\ voor te implementeren en verder onderzoek doen naar bestaande methodieken voor beheersen van datakwaliteit. Bijvoorbeeld door starten van een ``Data Stewardship Program" zoals omschreven wordt in het ``TDWI Report, The Data Warehouse Institute"\ door  \textcite{eckerson2002data}.

Spark SQL is bewezen snel erg snel te zijn en Amazon EMR is managed service

database column store heeft geen indexes maar partities, bewezen erg snel te zijn… maar geen voordelen worden behaald omdat  ons proces altijd N-max aantal columns worden opgevraagd. Daarom zal de query nooit sneller zijn, unless er meerdere query 's zijn waarbij verschillende columnen worden opgevraagd. Daarom zijn er te weinig voordelen om hier een POC mee te starten, daarnaast zullen versie upgrades lastig zijn. 

De Aggregatie processen met Golang zullen erg klein worden gemaakt door de data van te voren te partitioneren per shop. Dit biedt de mogelijkheid om het deze processen uit te voeren op verschillende machines (scale-out), door het gebruikt van resqueue workers "Resque workers can be distributed between multiple machines, support priorities, are resilient to memory bloat / "leaks," \parencite{github2016reque} 


Kort: Er wordt niet verder geïnvesteerd in een POC met databases omdat er te veel keuze is, en technologie niet effectief is toe te passen

Wel wordt er een POC uitgevoerd met Golang en Spark


% Het splitsen van data verwerken en een databases waaruit de data is op te vragen, heeft met gevolg dat Saki, een applicatie die in real time aggregaties op het hoogste niveau pakt (channel, shop, shop_product_id) en deze flexibel aggregeerd op basis van datum.
\clearpage

% ACTIVITY Bespreken van ontwerpen en evt. aanpassen van ontwerpen
% ACTIVITY Vastleggen kwaliteitscriteria Proof of concept's
\section{Ontworpen experimenten middels proof of concept}
\label{sec:deelvraag4}
Wat zijn de mogelijke oplossingen en hoe wordt dit gevalideerd?

In dit hoofdstuk wordt omschreven hoe de Proof of Concept 's op basis van Golang en Spark zijn uitgevoerd. Hieruit moet blijken of de technologieën die geselecteerd zijn in het onderzoek ook werkelijk toepasbaar zijn in de context.

Zoals wordt omschreven in de originele doelstelling in 3.1 moet data op tijd worden verwerkt en moet het tijdig herstellen van fouten mogelijk zijn. Dit experiment valideert welke oplossing de beste prestatie levert in de huidige situatie.

Het experiment bestaat uit twee fases. Eerst wordt er een Map fase uitgevoerd en vervolgens een Reduce, zoals bij MapReduce. De map fase is 
ETL en de reduce fase is een aggregatie met Golang of Spark SQL.

\subsection{Gebruikte methode}

Voor ieder POC worden er twee metingen uitgevoerd: de totale proces tijd, gemeten door GNU time \parencite{gnu_time}. En de totale tijd zonder de tijd tussen het starten van het proces en het werkelijk uitvoeren van de implementatie (ook wel warmup-time genoemd). Dit wordt in één serie vijf keer herhaald. Over de metingen wordt een gemiddelde genomen en dit vormt het resultaat.

Voor het evalueren van de twee POC 's is besloten om geen latency, IO, memory of CPU gerelateerde benchmarks te evalueren. Dit is wel gewoonlijk bij industrie benchmarks van verschillende data processing tools. Maar dit wordt gedaan vanuit een competitief perspectief  \parencite{ousterhout2015making}. Voor de doelstellingen van dit project is de totale proces tijd belangrijker.

Tijdens de drie fases van het experiment wordt gebruik gemaakt van een dataset uit productie als data input. De data set bestaat uit visits en orders van 264 webwinkels.

\begin{table}[h]
\centering
\caption{Gebruikte data bronnen met eigenschappen}
\label{tab:datasets}
\begin{tabular}{|l|l|l|l|l|l|}
\hline
data bron & data grootte voor & data grootte na & aantal regels voor & aantal regels na \\ \hline
orders.bson    &  0.2494 GB & 0.04734 GB       & N/A                & 31.869           \\ \hline
visits.json & 1.064 GB               & 0.3986 GB      & 2.345.341          & 2.345.090        \\ \hline
\end{tabular}
\end{table}

Tijdens het ETL proces is er sprake van data verlies van minder dan 1\%, zie \ref{tab:datasets}. Omdat dit minder is dan 1\% wordt dit niet als probleem gezien, daarnaast is er geen ruimte ingepland voor data validatie tijdens dit project.

\clearpage

\subsection{ETL}
\label{sec:etl}

ETL (Extract, Transform and Load) is een manier om data uit een externe databron te transformeren naar een ander schema \parencite{data-mining}. Dit zorgt voor een uniform formaat waardoor latere processen geen logica hoeven toe te passen om met de data te werken. In tabel \ref{tab:etl_input_example} is het dataformaat omschreven van de visits data bron. Iedere regel in dit bestand representeert een bezoek op de website van een webwinkel.

Tijden de ETL wordt de data geïnterpreteerd, zo wordt er bijvoorbeeld een kolom toegevoegd: \verb+Traffic=1+ voor iedere regel in visits.json. Dit komt ten goede aan de data consistentie omdat de logica in één fase plaatsvindt. De vervolg fases kunnen hierdoor gebruik maken van pure functies die in een framework of taal zijn geïmplementeerd.

\begin{table}[h]
\centering
\caption{Een voorbeeld van de input data gebruikt tijdens ETL}
\label{tab:etl_input_example}
\begin{tabular}{|l|l|l|l|l|l|}
\hline
timestamp  & time\_id & publisher\_id & shop\_id & shop\_product\_id & shop\_category\_id \\ \hline
1464154281 & 20160424 & 410         & 224      & 103774145         & 338790             \\ \hline
1464154306 & 20160424 & 61          & 910      & 108837485         & 6782117            \\ \hline
1464154314 & 20160424 & 410         & 224      & 100670758         & 9152995            \\ \hline
\end{tabular}
\end{table}


Het ETL proces is in de eerste iteratie van het project  geschreven in Golang. Hier is voor gekozen omdat er betere ondersteuning is voor data formaten zoals BSON, een binary formaat.

Als laatste is er voor gekozen om data op te splitsen per webwinkel als een methode van data partitionering. Dit biedt de mogelijkheid om in latere fases gebruik te maken van concurrency of parallellisatie implementaties, omdat data kan worden verwerkt in een proces per partitie. Uit onderzoek blijkt dat MapReduce deze methode gebruikt om data te schalen naar een ``terabyte-scale".

Na het verwerken van de input data worden er 530 bestanden weg geschreven. Voor iedere webwinkel bestaat er nu een bestand met de getransformeerde input data. Een voorbeeld van het datamodel is te zien in appendix \ref{app:json_format}. In het geval van de voorbeeld data set uit tabel \ref{tab:etl_input_example} zouden de volgende bestanden worden gecreëerd, zie figuur \ref{fig:exports_files}. 

\begin{figure}[h]
    \caption{Een lijst van bestanden die voor de voorbeeld data worden weggeschreven}
    \label{fig:exports_files}
\begin{verbatim}
./exports/1001/visits.json
./exports/1001/orders.json
./exports/224/visits.json
./exports/224/orders.json
\end{verbatim}
\end{figure}


\clearpage

\subsection{Golang}


De implementatie in Golang maakt gebruik van het concurrency model dat de porgrammeer taal aanbied. De implementatie is deels te lezen in appendix \ref{app:golang_code}. Iedere databron  wordt ingelezen in een aparte thread. In een andere thread wordt gebruikt om de data te ontvangen en voert de aggregatie functie SUM uit op alle kolommen. Alle data wordt gegroepeerd op basis van de key ShopID, PublisherID, en ProductID. Er zijn geen bijzonderheden gevonden waaruit blijkt dat de implementatie niet schaalbaar is in te zetten.

Om het proces te starten wordt de bestandslocatie van een partities gegeven. Bijvoorbeeld: \newline\verb+./bin/go_aggregator ./exports/1001/+

\subsection{Spark}

De implementatie in Spark maakt gebruik van een cluster dat in standalone mode is gestart. Er is 5 GB geheugen beschikbaar gesteld en 2 CPU cores. De implementatie is deels te lezen in appendix \ref{app:spark_code}. Iedere databron wordt ingelezen in het geheugen en met een Spark SQL query geaggregeerd.

Tijdens het experiment is gevonden dat de beste performance wordt behaald wanneer alle data in een groot bestand wordt gegeven. Spark berekend zelf de optimale partitionering voor de beschikbare hardware. Daarom wordt eerst iedere partities samengevoegd in één bestand.

Om het proces te starten voor Spark wordt het script gegevenaan het spark-submit commando. Bijvoorbeeld:\newline\verb+spark-submit query.py --input ./exports_combined.json+

In Spark is er wel sprake van warmup-time. Dit komt waarschijnlijk door de JVM (Java Virtual Machine) en de interpretatie van de DSL, namelijk Spark SQL.

\subsection{Benchmarks}

Voor de benchmars zijn uiteindelijk drie experimenten uitgevoerd per POC. Hierbij zijn product advertenties geaggregeerd over verschillende partitites, waarbij in test drie alle partities worden aggregeerde, met andere woorden de data voor alle 265 webwinkels.

\begin{figure}[h]
\caption{Omschrijving van de uitgevoerde tests}
\label{fig:test}
\begin{enumerate}
    \item data partitie van 68KB, dit wordt beschouwd als een kleine webwinkel.
    \item een partitie voor grote webwinkel met een data grootte van 9MB
    \item alle partities van alle shops met een data grootte van 387MB
\end{enumerate}
\end{figure}

In tabel \ref{tab:benchmarks} is af te lezen dat het verwerken van een partitie van 68KB gemiddeld 200 milliseconden duurt in Golang en 27 seconden met Spark. Voor iedere test omschreven in \ref{fig:test} is de totale tijd en de warmup-time vastgelegd. De totale tijd voor het ETL proces duurt gemiddeld 1 minuut en 40 seconden, en is niet opgenomen in de tabel.

\begin{table}[h]
\centering
\caption{Benchmark resultaten voor verschillende tests}
\label{tab:benchmarks}
\begin{tabular}{|l|l|l|l|l|l|l|}
\hline
\multirow{2}{*}{} & \multicolumn{2}{l|}{1e test} & \multicolumn{2}{l|}{2e test} & \multicolumn{2}{l|}{3e test} \\ \cline{2-7} 
                  & totale tijd   & warmup-time  & totale tijd   & warmup-time  & totale tijd   & warmup-time  \\ \hline
Golang            & 200ms         & N/A          & 1.4s          & N/A          & 1m en 22s     & N/A          \\ \hline
Spark total       & 27s           & 18s          & 37s           & 23s          & 3m en 22s     & 1m en 20s    \\ \hline
\end{tabular}
\end{table}

\clearpage

\subsection{Gebruikte Hardware}
\label{subsec:hardware_specs}

Tijdens het onderzoek naar verschillende technologieën is bewezen dat dat er voldoende schaalbaarheid kan worden bereikt door het gebruiken van "Commodity hardware". Hiermee worden machines bedoelt die algemeen beschikbaar zijn, en geen bijzondere performance voordelen bieden. Om in deze trend te blijven is er voor gekozen om geen speciale omgeving te installeren voor de experimenten. Uiteindelijk is er gebruik gemaakt van een Macbook Pro met de volgende specificaties:

\begin{table}[h]
\caption{Hardware specificaties van de gebruikte machine}
\label{tab:hardware_specs}
\begin{tabular}{ll}
Merk:      & MacBook Pro (Retina, 13-inch, Mid 2014) \\
Processor: & 2.6 GHz Intel Core i5                   \\
Geheugen:  & 8 GB 1600 MHz DDR3                      \\
Storage:   & Apple SSD                                  
\end{tabular}
\end{table}

\subsection{Conclusies}

In sectie \ref{subsec:3.3.2} werd gesteld dat: \textit{``In de nieuwe situatie moet op efficiënte wijze worden om gegaan met een kleine hoeveelheid aan data."}. In het experiment met Golang worden kleine partities van 68KB in 200 milliseconden verwerkt. In vergelijking met Spark is dit 27 seconden. Met deze resultaten wordt voorzien in de eerder gestelde eisen.

Opvallend is dat de warmup-time in Spark toeneemt op basis van de data input. In tabel \ref{tab:benchmarks} is af te lezen dat de implementatie in Golang altijd betere prestaties levert, ook wanneer alle webwinkels worden geaggregeerd. In het onderzoek werd al eerder geconstateerd dat mogelijke overhead een rol speelt in distributed systemen. De implementatie in Golang lijkt hier geen last van te hebben.


\clearpage

% ACTIVITY Uitvoeren van Proof of concept's
\section{Conclusies}
\label{sec:deelvraag5}
Wat zijn de gepresenteerde oplossingen en waarom zijn deze volledig of niet?


De volgende doelstelling:

Webwinkel eigenaren moeten in staat zijn om beslissingen te maken op basis van correcte en actuele gegevens in Adcurve. Dit betekent dat de gegevens die Adcurve toont altijd te verklaren zijn en overeenkomen met de werkelijkheid. De data moet op tijd verwerkt zijn, en fouten moeten tijdig hersteld kunnen worden.

HET ETL proces
% De snelheid waarmee de dataset wordt verwerkt in Golang is:
% 61.19s user 13.02s system 100\% cpu 1:13.75 total

% Als laatste biedt dit ETL ontwerp de mogelijkheid om de data voor een specifieke webshop opnieuw te verwerken. Dit betekend dat data eenmalig wordt gefilterd. 
Door de bestanden te bewaren op de hardeschijf voor de laatste 30 dagen wordt er nog meer voordeel behaald.
In het scenario wanneer er data correcties worden uitgevoerd voor één data bron op een specifieke datum, worden er minder bestanden verwerkt tijdens de ETL omdat de andere overige data bronnen al eerder zijn verwerkt. Dit betekend dat data eenmalig wordt gefilterd.



De Aggregatie processen met Golang zullen erg klein worden gemaakt door de data van te voren te partitioneren per shop. Dit biedt de mogelijkheid om het deze processen uit te voeren op verschillende machines (scale-out), door het gebruikt van resqueue workers "Resque workers can be distributed between multiple machines, support priorities, are resilient to memory bloat / "leaks," \parencite{github2016reque} 




\textbf{Go Aggregator}


per Shop:
average 1.42, 1.35, 1.40, 1.41, 1.32
1.38
output is 1 regel output en 401 bytes groot

per shop en publisher:
average 1.46, 1.35, 1.44, 1.41, 1.44
1.42
output is 27 regels output en 10640 bytes groot


per shop, publisher en advertentieproduct:
average 1.66, 1.51, 1.69, 1.59, 1.57
1.59 seconds is average time
output is 12982 regels 4722905 bytes


Omdat Golang compileert naar assembly is er geen sprake van warmup time. En er is daarom alleen de totale tijd.


Split all orders and visits for all shops (264 shops)

\begin{lstlisting}
2016/05/11 11:38:45 Begin
2016/05/11 11:40:09 Main Done
./start.sh  69.83s user 16.12s system 101% CPU 1:24.70 total

total time: 1 minute 9 seconds
\end{lstlisting}

\textbf{Spark all shop products, file per shop}

(one file contains all products for all channels)

\begin{lstlisting}
16/05/11 10:09:58 INFO SparkContext: Running Spark version 1.6.0
[2016-05-11 10:10:03] Start loading data
[2016-05-11 10:12:24] Finished writing data
Total time:     0:02:21.378717
16/05/11 10:12:24 INFO ShutdownHookManager: Deleting directory /private/var/folders/tp/….

time without job submit: 2 minutes 21 seconds
\end{lstlisting}


\textbf{Conclusie}

[todo]

Het gebruik van big data systemen is zeer schaalbaar, maar introduceerd een hoop overhead. (paper).

Voordelen van een programmeer taal is dat complexe business logic hierin kan worden geschreven zonder rekening te houden met veel abstracties zoals in parallel programming zoals map reduce of sql.. In zo'n paradigm moet een probleem naar code worden vertaalt met extra kennis naast het programmeren, kennis over het map reduce paradigm, sql join. Daarnaast wordt code vertaalt naar een query plan en het gebruik van indexes etc..


