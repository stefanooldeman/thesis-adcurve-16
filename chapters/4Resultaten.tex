\chapter{Resultaten}

In dit hoofdstuk worden alle resultaten uit het onderzoek gepresenteerd
Deelvraag 1 "Wat zijn de functionele en niet functionele eisen, waaraan de oplossing moet voldoen?" wordt beantwoord in \ref{sec:deelvraag1}

Deelvraag 2 "Wat zijn toonaangevende methodes en technologieën om statistieken te berekenen, en passen bij de wensen en eisen van de opdracht?"\  wordt beantwoord in \ref{sec:deelvraag2}

In \ref{sec:deelvraag3} wordt gekeken naar "Wat zijn de probleem scenario's waardoor gegevens niet te verklaren zijn, en wat zijn de mogelijke oplossingen?"\ 

In Deelvraag 4 is beantwoord in \ref{sec:deelvraag4} wordt belicht hoe de expirementen zijn ontworpen

Conclusies en een vergelijking wordt gemaakt in het betantwoorden van Deelvraag 5 "Wat zijn de gepresenteerde oplossingen en waarom zijn deze volledig of niet?" te lezen in \ref{sec:deelvraag5}

% MAIN QUESTION: "Hoe verzorgt een nieuwe implementatie voor het up-to-date houden van statistieken in Adcurve zodat gegevens altijd te verklaren zijn?"

\clearpage

% ACTIVITY Afleggen interviews rondom Requirements \& Constraints, documenteren use-cases
% RESULT MosCow prioriteiten lijst en checklist
\section{Functionele en niet functionele eisen}
\label{sec:deelvraag1}

Wat zijn de functionele en niet functionele eisen, waaraan de oplossing moet voldoen?

In een interview met Matthijs Jorissen zijn de functionele en niet functionele eisen besproken. Vervolgens is hierin een prioriteit aangebracht met behulp van de MoSCoW methode. Hierbij worden alle project eisen in vier groepen ingedeeld: priority groups “MUST have”, “SHOULD have”, “COULD have”, en “WON’T have” \textcite{ma2009effectiveness}.

\begin{comment}
Een functionele eis kan gezien worden als iets dat de gebruiker nodig heeft om het doel te bereiken of een bepaalde voorwaarde waaraan de oplossing moet voldoen.

Een non functionele eis is een beperking doe wordt opgelegd op een mogelijke oplossing, met het doel om functionele eisen te behalen of het doel van het project.
\end{comment}

\begin{table}[bh]
\centering
\caption{lijst van eisen geprioritiseerd met behulp van de MoSCoW analyse}
\label{table:requirements}
\def\arraystretch{1.5}

\begin{tabular}{|l|p{12.5cm}|}
\hline
\textbf{Prioriteit} & \textbf{Functionaliteit/Requirement}
\\ \hline
Must have           & Statistieken voor één webwinkel kunnen opnieuw worden gegenereerd zodat veranderingen in externe data bronnen, bijvoorbeeld orders of geïmporteerde kosten, opnieuw worden geaggregeerd. Dit moet kunnen tot 30 dagen terug.
\\ \hline
Must have           & Het creëren van data aggregaties voor alle webwinkels over een dag, mag niet meer tijd in beslag nemen dan de huidige oplossing nodig heeft. Dit is 1 uur en 30 minuten.
\\ \hline
Must have           & De kosten voor eventuele licenties en infrastructuur mogen niet hoger zijn dan dat voor de huidige oplossing.
\\ \hline
Should have         & Statistieken zijn altijd voor kantoor uren beschikbaar voor het dashboard en voor latere processen zoals het genereren van tips.
\\ \hline
Should have         & De gekozen oplossing heeft een productief programmeermodel en is afhankelijk van hardware architectuur.
\\ \hline
Could have         & De programmatuur is testbaar en hierdoor eenvoudig te onderhouden.
\\ \hline
Could have          & Data aggregaties voor een groep webwinkels kunnen op een ander tijdstip worden geaggregeerd i.v.m. verschillende tijdszones.
\\ \hline
Could have         & De nieuwe oplossing moet schaalbaar zijn tot 10.000 webshops.
\\ \hline
\end{tabular}
\end{table}

Een aantal gebruikte termen zoals simpel, snel etc. worden verduidelijkt met de bijbehorende definities die in overleg tot stand zijn gekomen:

\begin{itemize}
    \item \textbf{Een productief programmeermodel} in de context van dit project wordt omschreven in \textcite{asanovic2006landscape}: "programming models should be independent of the number of processors, they should allow programmers to use a richer set of data types and sizes, and they should support successful and well-known parallel models of parallelism"

    \item \textbf{Scalability} wordt door \textcite{dubey2005recognition} gedefinieerd als: Schaalbaarheid is het vermogen om de gewenste snelheid te bieden en te onderhouden al dan niet verbeteren. De industrie heeft hier twee manieren voor. ``Scale up"\ is de methode waarbij hardware wordt vervangen zodat betere prestatie geleverd kan worden. Bij ``scale out"\ wordt er extra hardware aangesloten zodat een bestaand systeem de toenemende werkdruk kan ondersteunen.
\end{itemize}

\begin{comment}
TODO
\begin{itemize}
    \item Data aggregaties moeten plaatsvinden zodra kosten beschikbaar zijn per publisher
    
    \item voor probleem scenario 1 en 4, moet mogelijk zijn om voor een individuele webwinkels de data te aggregeren zodat er correcties kunnen worden gemaakt bij data kwaliteit issues. 
    \item de snelheid waarmee data wordt verwerkt moet snel genoeg zijn om 30 dagen aan data binnen 1 dag te verwerken.
    
    \item De tijd die het kost voor data aggregaties moet voorspelbaar zijn, dit kan worden bereikt wanneer de oplossing lineair schaalbaar is.
\end{itemize}

\end{comment}




\clearpage

% ACTIVITY Onderzoeken van toonaangevende, passende technologieën
\section{Toonaangevende methodes en technologieën}
\label{sec:deelvraag2}

Wat zijn toonaangevende methodes en technologieën om statistieken te berekenen, en passen bij de wensen en eisen van de opdracht?

Uit verschillende bronnen zijn richtingen gevonden voor mogelijke oplossingen. Zoals het gebruik van database oplossingen, distributed oplossingen, en hardware oplossingen.
\footnote{
Uit de volgende opgevraagde webpagina's zijn verschillende onderzoeksrichtingen gevonden. De github pagina heeft over 200 technologieën en referenties naar verschillende papers.
\begin{enumerate}
    \item \url{https://github.com/onurakpolat/awesome-bigdata}
    \item \url{https://experfy.com/blog/hadoop-market-size-adoption-growth-2020/}
    \item \url{https://cs.arizona.edu/~bkmoon/papers/sigmodrec11.pdf}
    \item \url{http://datanami.com/2014/11/21/spark-just-passed-hadoop-popularity-web-heres/}
    \item \url{http://radar.oreilly.com/2014/01/a-compelling-family-of-dsls-for-data-science.html}
\end{enumerate}
}


\subsection{Databasese oplossingen}

De volgende lijst van database systemen beschikken over analytical functionaliteiten nodig voor het aggregeren en creeëren van statistieken. Voor het maken van data aggregaties zijn baisis SQL functies nodig: SUM, MIN, MAX, AVERAGE, samen met de SQL clauses: GROUP BY en HAVING. \parencite{data-mining} Omdat bijna alle database systemen deze functies ondersteunen is er gekeken naar databases die een hoog volume aan data kunnen verwerken. 
Omdat er nu per dag zo'n 1.8Gb aan data voor alleen visits wordt verwerkt en dit extropaleert naar (1.8/480) * 100000 = 375Gb per dag.

Traditionele databases systemen zijn gemaakt om transactionele queries te beantwoorden. Deze zullen minder schaalbaar zijn doordat er verschillende fases zijn die overhead introduceren. \parencite{harizopoulos2008oltp}
Daarnaast introduceert het frequent updaten en corrigeren data bronnen mogelijke performance problemen doordat data consistentie wordt gegarandeerd met methodes als "table locks". Verder raken databases overbelast wanneer de opgravraagde database query resulteert in een dataset die nit in het geheugen is te plaatsen.  daarom tot een trager, overbelast systeem. \parencite{kersten2011researcher}

Database queries kunnen overigens erg traag raken wanneer databases groeien en query resultaten niet meer in RAM passen. Hierbij wordt resulteert een terugval op de hardeschijf (zogeheten "Disk spills") in het vertragen van een datapipeline. Ook zijn er verschillende anakedotes vanuit de organisatie waarbij databases zoals MySQL sporadisch, overbelast raakt.

%"Moreover, database queries often contain blocking operations that lead to a pipeline stall or spilling large intermediates back to the disks." \parencite{kersten2011researcher}


Het plan in onze use-case is om frequent data aggregaties uit te voeren zodat statestieken in adcurve worden gecorrigeerd. Dit kan er uiteindelijk voor zorgen dat locking systemen om data consistentie te garanderen een bottle neck op a 
Traditionele databases garanderen waarbij een transactionele niet worden beschouwd als een verkiesbare technologie. Daarom is gekeken naar databases met een andere storage engine.

Opvallend is dat bijna alle distributed en analytics databases van PostgreSQL afstammen \parencite{postgresforks}

Hieruit valt te concluderen dat de business intelligence industry veel research heeft geinvesteerd in op basis van Postgress technology.

Michael Stonebraker is origineel auteur Postgress en later vele high performance databases. In zijn paper "How To Build a High-Performance Data Warehouse"\  omschrijft hij drie verschillende architectuur typen / storage engines: Shared Memory, Shared Disk en Shared Nothing.
"Because shared nothing does not typically have nearly as severe bus or resource contention as shared-memory or shared-disk machines, shared nothing can be made to scale to hundreds or even thousands
of machines. Because of this, it is generally regarded as the best-scaling architecture". \parencite{dewitt2006build}



In de paper worden de volgende databases genoemd:
\begin{table}[bh]
\centering
\caption{My caption}
\label{my-label}
\begin{tabular}{|l|l|l|}
\hline
\textbf{Shared memory} & \textbf{Shared Disk} & \textbf{Shared Nothing} \\ \hline
Microsoft SQL Server   & Oracle RAC           & Teradata \\
\hline
PostgresSQL            & Sysbase IQ           & Netezza \\
\hline
MySQL                  &                      & IBM DB2 \\
\hline
                       &                      & GreenplumDB \\
\hline
                       &                      & Vertica \\
\hline
\end{tabular}
\end{table}

\subsubsection{\textbf{Conclusie}}

TODO: Vorm een conclusie over databases die bij de requirements passen.
\clearpage

\subsection{Distributed oplossingen}
% Wanneer grootte hoeveelheden data worden verwerkt is een goed alternatief "Joe Hellerstein, our local expert in databases, said the future of databases was large data collections typically found on the Internet. A key primitive to explore such collections is MapReduce" \parencite{asanovic2006landscape}, 


- voor het verwerken van grootte hoevelheden data
- toepassingen / reden van map reduce
- toepassingen / reden van Spark
- andere ervaringen

Verdere mogelijke technologieën zijn alternatieven 
De huidige oplossing maakt gebruik van MongoDB die een MapReduce framework heeft geimplementeerd.


Doordat de verhouding tussen hardware capiciteit en de kosten hiervan steeds gunstiger werden en de toenemende hoeveelheid data die gegenereerd wordt door het internet..

Voor het verwerken van de toenemende hoeveelheid data gegenereerd door  internet
"Joe Hellerstein, our local expert in databases, said the future of databases was large data
collections typically found on the Internet. A key primitive to explore such collections is
MapReduce, developed and widely used at Google \parencite{dean2008mapreduce}" \parencite{asanovic2006landscape}, 

Het alternatief naast databases zijn MapReduce frameworks, waarbij data wordt opgeslagen en beheerd in een cluster, aldus een distributed systeem. Veel van de populaire 

Voor het verwerken van grootte hoeveelheden data werden distributed systemen bij Google succesvol ingezet, 
Skeptisch, databases niet geschikt voor het verwerken van grootte hoeveelheden data. 

Distributed systemen zijn in 2006 geintroduceerd door google met het GFS en MapReduce. \parencite{dean2008mapreduce}
In \textcite{asanovic2006landscape} werd MapReduce nog als het alternatief gezien voor het verwerken van grootte datasets. En hoewel databases later ook distributed systemen werden (Shared nothing, column stores zijn nl distributed), en dit tot Petabyte schaal kan worden ingezet. \parencite{lamb2012vertica}
Is het vele malen duurder!
De populaire MapReduce platformen zijn nog altijd te configureren met "Commodity Hardware" etc. \parencite{dean2008mapreduce}


Voor langere tijd nu, staat Hadoop bekende als een volwasse technologie en implementatie van Map Reduce, zo schrijft Gartner in een recent rapport: "The open-source software framework known as Apache Hadoop has gained sizable acceptance in organizations, spurred on by the growing digital business appetite for big data" \parencite{hadoop2013selection}.

Vele organisaties weten echter nog niet hoe ze deze technologie effectief kunnen inzetten. OOk al hebben organisaties nu wel de kennis, blijft de adoptie nog altijd laag. \parencite{hadoop2015adoption}


\begin{comment}
De evolutie van distributed applictions Google File System -> Map Reduce -> Apache Hadoop -> Hype is groot maar Mainstream Adoption is nog erg laag, complex landschap. De organisatie heeft eerder een Proof Of Concept gerealiseerd maar zag op tegen het beheer van een cluster. 
Tegenwoordig is er ook success met hadoop in de cloud, aangeboden door Amazon EMR.
Minpunten:
- Hoge kosten
- Opstart tijd van 17 minuten (zie benchmarks Spark vs Hadoop)

Alternatief Spark erg enthasiaust vanuit het development Team, ook mogelijk op Amazon EMR
\end{comment}

% Rond 2009 MapReduce werd geïdentificeerd als een "Prominent parallel data processing tool" dat tractie kreeg vanuit zowel de industrie als de academische wereld. Zo concludeerde zij dat: "Naast traditionele Database Management Systems is Map Reduce een zeer schaalbare en flexibele oplossing is voor een variatie aan data analyses". \parencite{aarnio2009parallel} 


% Kort nadat UC Berkley zijn cluster framework "Spark" heeft gedoneerd aan de Apache foundation in 2014, heeft het project enorme aandacht ontvangen vanuit de hele industrie. \parencite{spark2015intro}. Zo is er ook vanuit Shop2market enorme interesse in Apache Spark. Er zijn al eerder in 2015 Proof of Concept uitgevoerd met betrekking tot machine learning applicaties.

% Terwijl nog veel organisaties zweven tussen het begrijpen en het vinden van de toepassing voor Hadoop \parencite{hadoop2015adoption}, is sinds 2013 een alternatief in memory implementatie van o.a. Map reduce enorm populair  \parencite{spark2015intro}

\subsubsection{\textbf{Conclusie}}

De uiteindelijke selectie voor dit onderzoek is beperkt. De frameworks die zijn geselecteerd hebben genoeg herkenning vanuit de organisatie en industrie. De enorme lijst op github is getuige van de enorme hype rondom Big Data. Toch zijn veel organisaties nog zoekend naar wat technologie betekent en hoe het waarde toevoegd\parencite{hadoop2015adoption}.

Verder zijn er in de eerder genoemde lijst (github) enorm veel tools benoemd. In dit onderzoek zullen naast Hadoop en Naast eerder besproken frameworks Spark en Hadoop zijn er andere technologieën waar colleg's eerdere ervaring mee hebben.

De selectie van technologieën komt uit op

\begin{itemize}
    \item Disco
    \item Apache Flink
    \item Hadoop (Apache MapReduce)
    \item Apache Spark
\end{itemize}

Streaming methodes zoals bijvoorbeerld Apache Storm worden als toonaangeevend gezien maar zijn niet geselecteerd. Dit omdat de vorm waarin de data beschikbaar wordt gesteld aan het aggregatie proces altijd in batch is. Het valt buiten de scope om deze data flows te veranderen.
\clearpage

\subsection{Hardware oplossingen}
High performance Domain Specific Languages zijn veelbelovend.

http://reconfigurablecomputing4themasses.net/files/1.2%20Kunle.pdf

Verilog (VHDL)
CUDA
OpenCL
Threads in (Java C++)
OpenMP

Zijn embedded programmeer talen die wel maximaal gebruik kunnen maken van hardware capiciteit. Echter is het programmer model niet vriendelijk en niet compatible met ieder platform.



Het programmeren van hardware is mogelijk met talen zoals CUDA, OpenCL en OpenMP. Deze worden in de rest van deze thesis benoemd als hardware frameworks. Echter zijn dit geen geschikte kandidaten omdat geen geschikt programmeer model bieden. Zo wordt wordt herkend door de Machine learning researchers van (Standfort University) \cite{sujeeth2011optiml} dat deze talen niet productief genoeg zijn om effectief problemen op te lossen en maximaal gebruik te maken van de hardware componenten zoals CPU en GPU. Dit komt omdat de programmeur direct te maken krijgt met complexiteit van het paralliseren van functies en moet tevens rekening houden met hardware specifieken uitdagingen. \parencite{chafi2010language} \\

Terwijl \cite{sujeeth2011optiml} OptiML presenteerd als een alternatief is het project\footnote{Zie de website van OptiML: https://stanford-ppl.github.io/Delite/optiml/index.html} nog altijd in Alpha versie. Echter is in 2014 door hetzelfde team Forge gepresenteerd. Een DSL voor hardware frameworks waar de test resultaten erg veelbelovend zijn. "Forge-generated Delite DSLs perform within 2x of hand-optimized C++ and up to 40x better than Spark, an alternative high-level distributed programming environment." \parencite{sujeeth2014forge}


Naast de gevonden technologieën wordt binnen shop2market veel gebruik gemaakt van Golang. (verdere uitleg nodig)
Vanuit de organisatie zijn hier in 2014 Proof Of Concepts mee gedaan waarin het concurrency model en performance is vergelijken met andere Akka (scala), Erlang en Java. Hierin waren Java en Golang even snel, echter is Golang een moderne programmeer taal waarin het concurrency model anders is geimplementeerd. Hierdoor hoeft geen rekening te houden met Threads.

Rust

\subsubsection{Conclusie}

De selectie van technologieën komt uit op

\begin{itemize}
    \item Forge
    \item Golang
\end{itemize}


\subsection{Conclusies}

Er zijn een aantal oplossings richtingen gevonden:
- clustered databases analytics / datawarehouses (RDBMS): google bigquery, vertica, redshift, hadoopDb, Teradata
- map reduce platform: spark, hadoop, disco etc.
- Hardware, software solutions: Forge en Golang Python, MATLAB, R, Golang, Rust


\clearpage

% ACTIVITY Impact analyse uitvoeren rondom use-cases
\section{Problematische scenario 's en mogelijke oplossingen}
\label{sec:deelvraag3}
\textit{Wat zijn de probleem scenario's waardoor gegevens niet te verklaren zijn, en wat zijn de mogelijke oplossingen?}

Tijdens de project analyse is met verschillende stakeholders binnen de organisatie gesproken. Naast de gedefinieerde project eisen, eerder omschreven in tabel \ref{table:requirements} zijn er verschillende symptomen verzameld. Op basis hiervan is een Impact analyse gemaakt, waarbij de ernst van de situatie wordt uitgedrukt in een waarde. Zie hiervoor bijlage (TODO ref).
% TODO leg uit hoe de waarde tot stant komt FEMA

Vervolgens is per probleem scenario een mogelijke technische verklaring gezocht, zijn er criteria omschreven waaraan de mogelijk oplossing moet voldoen. Hierdoor wordt getoetst of de geselecteerde technologieën in  \ref{sec:gevonden_tools} toepasbaar zijn en de huidige probleem scenario werkelijk kunnen oplossen.

\subsection{Scenario's uit de Impact Analyse}


\begin{enumerate}
    \item L6: Api's provided by the publisher are not available
    \item M5: Apis' for google,change and the import stops working, or fail silent (values are not matched,,resulting in empty fields)
    \item H6: Wrong costs by configuration, issues: mistakes in tracking Order amount, tracked wrong, multiplied by 100 bug
    \item L7: Metrics calculated incorrectly because of mismatching product data (ids from tracking don't match cost exports from publisher)
    \item L7: Calculations fail,because technology or infrastructure 
    \item L4: Aggregations from different dimensions on the same data
    don't match
    \item H7 Bij het bekijken van dashboards in Adcurve zijn de gegevens over de vorige dag pas beschikbaar na 12:30 CET. Omdat alle data in een keer wordt verwerkt wordt het proces gestart nadat alle afhankelijke data bronnen beschikbaar zijn.
\end{enumerate}

Scenario een (1) tot en met vier (4) zijn niet te voorkomen omdat de organisatie zelf hier weinig invloed op kan hebben. Het wordt veroorzaakt door externe processen bij publishers, of mensen in diesnt van de webwinkel die de advertenties of webwinkel integraties verkeerd configureren. Scenario vijf (5) en zes (6) hebben een hoge impact en zijn specifieke problemen met de huidige implementatie. Deze scenario's zijn te voorkomen door het inzetten van andere technologie of een andere strategie als het aan komt op data processing.
Scenario 7 is eerder besproken tijdens de non functionele eisen in \ref{table:requirements}. Deze is opgenomen zodat onderzocht wordt hoe dit het best kan worden voorkomen.  

\clearpage

\subsection{Technische te verklaren oorzaken}

Met betrekking tot scenario vijf (5) en zeven (6) zijn verschillende problemen gevonden in relatie tot de gebruikte versie van MongoDB \parencite{mongo_changelog}. Uit monitoring systemen zijn de foutmeldingen terug te relateren aan software bugs die aanwezige zijn in versie 2.4.

De consensus binnen de organisatie is dat een upgrade naar een volgende versie riskant is. Het is niet in te schatten hoe veel problemen er zullen optreden. Daarnaast moeten er processen worden ingericht om het synchroniseren van order te ondersteunen en data verlies te voorkomen wanneer de database offline moet.
% TODO: is het mogelijke om machines van verschillende versies naast elkaar te draaien in een cluster?
    
Met betrekking tot scenario zeven (7) zijn de volgende mogelijke oorzaken gevonden:

\begin{itemize}
    \item Uit monitoring systemen blijkt dat het online productie processen van invloed zijn op de performance van het aggregatie proces. Omdat MongoDB direct wordt gebruikt voor het synchroniseren van bestellingen met webwinkels, en dit een realtime proces is, wordt het cluster hiermee belast. In de huidige situatie worden de MapReduce jobs op een aantal machines in het cluster uitgevoerd, toch word de performance beïnvloed.
    \item het aggregeren van data kan in MongoDB wellicht trager zijn dan nodig omdat de tussentijdse resultaten uit de fases in MapReduce worden weggeschreven naar verschillende machines \parencite{mongo_mr_shards}. Zo valt ook af te lezen uit monitoring dat het wegschrijven van de uiteindelijk resultaten langer duurt dan het aggregeren zelf. Dit heeft mogelijk te maken met het bijhouden van indexes.  \parencite{mongo_write_performance}
    % replica sets? sharding? grootte cluster?
\end{itemize}


\subsubsection{\textbf{Conclusie}}
\label{subsec:3.3.2}

Tijdens de root cause analysis zijn een aantal oorzaken omschreven. Hieruit valt te concluderen dat een mogelijke nieuwe oplossing aan de volgende criterea moet voldoen:

\begin{enumerate}[label=(\alph*)]

    % Er rekening moet worden gehouden met de mogelijke overhead van networking applications, en tegelijkertijd de mogelijkheid moet zijn om grootte data sets op te splitsen zodat er geen ``disk spills"\ optreden.
    \item In de nieuwe situatie moet op efficiënte wijze worden om gegaan met een kleine hoeveelheid aan data, zodat de verschillende fases van het aggregeren (zoals map, reduce, shuffle en sort). Dit omdat voor 80\% van de webshops een kleine hoeveelheid data wordt verzameld. Tegelijkertijd is de input data set groot en zijn resultaten van aggregaties wellicht groter. Er moet in deze situatie op efficiënte wijze data worden weg geschreven naar een eindlocatie,  zoals het filesysteem. Hierbij moet het onderhouden van indexes worden voorkomen.
    
    \item Er moet situaties situatie worden voorkomen waarin het onmogelijk is om een technologie of systeem te upgraden naar een nieuwere versie. Dit is te voorkomen door a) geen state te bewaren in het systeem zelf, maar dit alleen te gebruiken bij data processing. b) door het systeem onafhankelijk te maken van productie / online processen. Dit biedt de mogelijkheid door bijvoorbeeld een database backup te doen, het systeem te migreren en de data opnieuw in te laden.
\end{enumerate}

\clearpage


\subsection{Hoe wordt dit voorkomen met de gevonden tools / POC?}
\label{subsec:deelvraag3_vergelijking}


\subsubsection{\textbf{Versie upgrades}}

Door de tools binnen het Hadoop ecosysteem wordt geen state bewaard. Maar alle data wordt beheerd door HDFS en opgeslagen als toegankelijke bestanden in een gekozen bestandsformaat. Het risico is dat het upgraden van versies, niet compatible met verschillende bestands formaten. Ook de tools die opereren op het data formaat moetn compatible zij. Daarnaast bestaat de mogelijkheid dat HDFS data anders beheerd onder verschillende versies. Hierdoor zijn data migraties erg operationeel intensief is. Hierdoor is het sterk aan te raden om dit niet zelf te beheren, maar gebruik te maken van distributie zoals Amazon EMR, Hortonworks, Cloudera of MapR.

Omdat Spark onder andere uitmaakt van het Hadoop ecosysteem zijn deze conclusies te generaliseren.

Omdat databases bij definitie statefull zijn, moet er in dit geval een backup en restore mogelijkeid aanwezig zijn om upgrades te kunnen uitvoeren.

In het geval van Golang, dat in het algemeen niet als data processing tool te beschouwen is maar als een General purpose language, is de implementatie code eenvoudig te upgraden naar nieuwere versies. Dit wordt gegarandeerd als de implementatie volledig getest is door unit tests. 

\subsubsection{\textbf{Mogelijke overhead bij write operations}}
% ====================================================
% ====================================================

% USE THE COST PAPER and 

% https://www.usenix.org/conference/hotos15/workshop-program/presentation/mcsherry

% USE THE PAPER MEASURING NETWORK OVERHEAD IN DATA PROCESSING TOOLS VS CPU
% https://www.usenix.org/system/files/conference/nsdi15/nsdi15-paper-ousterhout.pdf

% REFERENCE TO THE PAPER DESCRIBING THAT:
% SPARK USES NO INDEX
% MAP REDUCE (Google / Apache) USES NO INDEX
% SHARED NOTHING DATABASES USE NO INDEXES

% ALSO APPLY Amdahls LAW
% https://en.wikipedia.org/wiki/Amdahl%27s_law

%  AND MAYBE 
% roughly 80\% of the effects come from 20\%
% https://en.wikipedia.org/wiki/Pareto_principle

% ====================================================
% ====================================================
    
\subsection{Conclusie}

\begin{comment}

% ``De huidige oplossing is niet liniar-scalable", is algemene idee binnen de organisatie. 

a) Een deel van het probleem is het actief monitoren van de data kwaliteit en processen starten om het te corrigeren. Hier wordt niet verder op in gegaan omdat een dergelijke oplossing hiervoor buiten scope valt.

b) Hiervoor geldt, zodra er een oplossing is moeten de datakwaliteit worden herstelt door het proces te starten dat de correcties uitvoert [``effected datasources"\ opnieuw evalueren]
Deze scenario's zijn ook tegelijkertijd lastig te voorkomen, alleen te herstellen. Het advies is daarom om hier ``monitoring en alerting\ voor te implementeren. Zie ook andere maatregelen zoals bijv. het aanwijzen van Data stewards (TODO: reference term, paper)


In deelvraag 2 in sectie \ref{sec:deelvraag2} is een shortlist gepresenteerd van geschikte technologieën voor.

- conclusie
    - data pipeline design
        - oplossing om  high query speed en good analytics performance te behalen met  twee losse systemen
        - S3 als datawarehouse storage ipv een database (geen limitations)
    - query speed wordt opgelost door extern systeem (Saki)
    - nieuwe systeeem voor mapreduce hebben de bestaande problemen niet omdat:
        - netwerken voorkomen wordt, workers zijn bewezen duizenden jobs aan te kunnen (kleine processen)
        - spark in memory oplossing is bewezen snel te zijn en EMR is managed service
        - (database column store heeft geen indexes maar partities, bewezen erg snel te zijn… maar geen voordelen worden behaald omdat  ons proces altijd N-max aantal columns wordt gebruikt opgevraagd)
\end{comment}
\clearpage


% ACTIVITY Bespreken van ontwerpen en evt. aanpassen van ontwerpen
% ACTIVITY Vastleggen kwaliteitscriteria Proof of concept's
\section{Ontworpen experimenten middels proof of concept}
\label{sec:deelvraag4}
Wat zijn de mogelijke oplossingen en hoe wordt dit gevalideerd?

Alle mogelijke oplossingen zijn besproken met het team, hierin is besloten dat er 3 mogelijke POC's worden uitgevoerd met de volgende technologieën:

\begin{itemize}
    \item Golang
    \item Apache Spark
\end{itemize}

\textbf{Databases} worden niet overwogen vanwege het programmeermodel, deze is zeer productief maar is niet eenvoudig te testen. Daarnaast heeft de organisatie niet de expertise om een Shared nothing database systeem (een cluster van machines) in een productie omgeving te onderhouden en te schalen. \\

\textbf{Parallel programming} wordt gebruikt in het POC met Spark. Door gebruik te maken van distributed functionaliteiten van Spark wordt getest of de snelheid waar spark bekend om staat (paper) behaald kan worden in onze use case. Er is de mogelijkheid om unit tests en functional tests te schrijven, dit wordt aangeboden door het framework.
het wegschrijven van meerdere resultaten in een proces kan door het partitionener van output data op (PublisherID, ShopID). Ook kunnen er meerdere dagen in een proces worden verwerkt en het resultaat wordt weggeschreven per TimeID, PublisherID en ShopID
\\

Als \textbf{Hardware oplossing} wordt gebruik gemaakt van Golang. Hierbij zal de performance van multicore processors worden getest. Deze taal is vergelijkbaar met C++ in zijn performance. Uit test resultaten door \cite{lee2010debunking} is gevonden dat een MapReduce operaties (Monte Carlo algoritme) sneller is op een CPU in vergelijking met een GPU (Te programmeren via eerder besproken andere DSL) "We typically find that the highest performance is achieved when multiple threads are used per core. For Core i7, the best performance comes from running 8 threads on 4 cores" \parencite{lee2010debunking} \\

\textbf{De volgende kwaliteits eisen zijn vastgelegd aan ieder POC} \\

TODO

% De aanroep van het process moet data uit de volgende datasets uit s3 lezen:
% de visits (csv) dump uit Cassandra, orders (bson) dump uit MongoDb, Geïmporteerde advertentiekosten van diverse publishers (csv)
\clearpage

% ACTIVITY Uitvoeren van Proof of concept's
\section{deelvraag 5 - Conclusies}
\label{sec:deelvraag5}
Wat zijn de gepresenteerde oplossingen en waarom zijn deze volledig of niet?

\textbf{Go Aggregator}
Split all orders and visits for all shops (264 shops)

\begin{lstlisting}
2016/05/11 11:38:45 Begin
2016/05/11 11:40:09 Main Done
./start.sh  69.83s user 16.12s system 101% cpu 1:24.70 total

total time: 1 minute 9 seconds
\end{lstlisting}

\textbf{Spark all shop products, file per shop}

(one file contains all products for all channels)

\begin{lstlisting}
16/05/11 10:09:58 INFO SparkContext: Running Spark version 1.6.0
[2016-05-11 10:10:03] Start loading data
[2016-05-11 10:12:24] Finished writing data
Total time:     0:02:21.378717
16/05/11 10:12:24 INFO ShutdownHookManager: Deleting directory /private/var/folders/tp/….

time without job submit: 2 minutes 21 seconds
\end{lstlisting}


\textbf{Conclusie}

[todo]

Het gebruik van big data systemen is zeer schaalbaar, maar introduceerd een hoop overhead. (paper).

Voordelen van een programmeer taal is dat complexe business logic hierin kan worden geschreven zonder rekening te houden met veel abstracties zoals in parallel programming zoals map reduce of sql.. In zo'n paradigm moet een probleem naar code worden vertaalt met extra kennis naast het programmeren, kennis over het map reduce paradigm, sql join. Daarnaast wordt code vertaalt naar een query plan en het gebruik van indexes etc..

