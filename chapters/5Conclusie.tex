\chapter{Afsluitend}
\label{ch:afsluitend}

In dit onderzoek hebben de verschillende fases bijgedragen aan het beantwoorden van de volgende hoofdvraag: \textit{Hoe verzorgt een nieuwe implementatie voor het up-to-date houden van statistieken in Adcurve zodat gegevens altijd te verklaren zijn?}.

Tijdens het project zijn vele technologieën onderzocht en gevalideerd. Uit het onderzoek is gekomen dat de tweetal technologieën die al bekend zijn binnen de organisatie perfect toepasbaar zijn voor het gegeven probleem. In \ref{sec:deelvraag5} is geconcludeerd dat de beste prestaties worden behaald met een implementatie in Golang.

De hoofdvraag wordt beantwoord doordat de gevalideerde oplossing voldoen aan alle gedefinieerde projecteisen. Daarnaast zijn er adviezen die bijdragen aan de beantwoording van de hoofdvraag:

In de huidige situatie worden de data aggregaties op één tijdstip uitgevoerd. Het advies is om dit proces anders in te richten, namelijk:
Groepeer alle webwinkels per publisher, en verwerk iedere databron direct nadat deze beschikbaar wordt gesteld. Dit betekend dat niet alle dashboards in Adcurve altijd up-to-date zullen zijn, maar dit is afhankelijk van de publisher. Dit moet worden gecommuniceerd bijvoorbeeld bij het zien van een dashboard. Hierdoor wordt het proces transparant voor de gebruikers, en zullen vertragingen te verklaren zijn.

Verklaarbaarheid duidt ook op het corrigeren van gegevens die incorrect zijn. Het advies is daarom om een proces in te richten waardoor incorrecte gegevens tijdig worden gedetecteerd. In de visie van de student is het mogelijk om data correcties automatisch uit te voeren nadat een databron is hersteld. Dit moet echter wel worden onderzocht en zou de organisatie ondersteunen bij de gewenste groei. Daarom is het goed om per integratie een systeem in te richten waarbij het ontbreken van data wordt gedetecteerd, en na het herstel daarvan aggregaties opnieuw worden gestart. 

Op deze wijze wordt het tweede deel van de hoofdvraag beantwoord en zullen statistieken in Adcurve altijd te verklaren zijn. De gegeven opdracht was om een Proof of Concept te ontwikkelen. Hierbij is de volgende doelstelling gebruikt:

\textit{Webwinkel eigenaren moeten in staat zijn om beslissingen te maken op basis van correcte en actuele gegevens in Adcurve. Dit betekent dat de gegevens die Adcurve toont altijd te verklaren zijn en overeenkomen met de werkelijkheid. De data moet op tijd verwerkt zijn, en fouten moeten tijdig hersteld kunnen worden.}

De student is er van overtuigt dat een vervolg project met de gegeven doelstelling met succes kan worden afgerond. Hierbij kunnen de adviezen worden gebruikt in het wijzigen van huidige processen, en de gepresenteerd methodes kunnen worden gebruikt voor het berekenen van statistieken.