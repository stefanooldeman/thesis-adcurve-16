
Wat zijn de functionele en niet functionele eisen, waaraan de oplossing moet voldoen?

In een interview met Matthijs Jorissen zijn de functionele en niet functionele eisen besproken. Vervolgens is hierin een prioriteit aangebracht met behulp van de MoSCoW methode. Hierbij worden alle project eisen in vier groepen ingedeeld: priority groups “MUST have”, “SHOULD have”, “COULD have”, en “WON’T have” \parencite{ma2009effectiveness}.

\begin{comment}
Een functionele eis kan gezien worden als iets dat de gebruiker nodig heeft om het doel te bereiken of een bepaalde voorwaarde waaraan de oplossing moet voldoen.

Een non functionele eis is een beperking doe wordt opgelegd op een mogelijke oplossing, met het doel om functionele eisen te behalen of het doel van het project.
\end{comment}

\begin{table}[bh]
\centering
\caption{lijst van eisen geprioritiseerd met behulp van de MoSCoW analyse}
\label{table:requirements}
\def\arraystretch{1.5}

\begin{tabular}{|l|p{12.5cm}|}
\hline
\textbf{Prioriteit} & \textbf{Functionaliteit/Requirement}
\\ \hline
Must have           & Statistieken voor één webwinkel kunnen opnieuw worden gegenereerd zodat veranderingen in externe data bronnen, bijvoorbeeld orders of geïmporteerde kosten, opnieuw worden geaggregeerd. Dit moet kunnen tot 30 dagen terug.
\\ \hline
Must have           & Het creëren van data aggregaties voor alle webwinkels over een dag, mag niet meer tijd in beslag nemen dan de huidige oplossing nodig heeft. Dit is 1 uur en 30 minuten.
\\ \hline
Must have           & De kosten voor eventuele licenties en infrastructuur mogen niet hoger zijn dan dat voor de huidige oplossing.
\\ \hline
Should have         & Statistieken zijn altijd voor kantoor uren beschikbaar voor het dashboard en voor latere processen zoals het genereren van tips.
\\ \hline
Should have         & De gekozen oplossing heeft een productief programmeermodel en is afhankelijk van hardware architectuur.
\\ \hline
Could have         & De programmatuur is testbaar en hierdoor eenvoudig te onderhouden.
\\ \hline
Could have          & Data aggregaties voor een groep webwinkels kunnen op een ander tijdstip worden geaggregeerd i.v.m. verschillende tijdszones.
\\ \hline
Could have         & De nieuwe oplossing moet schaalbaar zijn tot 10.000 webshops.
\\ \hline
\end{tabular}
\end{table}

Een aantal gebruikte termen zoals simpel, snel etc. worden verduidelijkt met de bijbehorende definities die in overleg tot stand zijn gekomen:

\begin{itemize}
    \item \textbf{Een productief programmeermodel} in de context van dit project wordt omschreven in \textcite{asanovic2006landscape}: "programming models should be independent of the number of processors, they should allow programmers to use a richer set of data types and sizes, and they should support successful and well-known parallel models of parallelism"

    \item \textbf{Scalability} wordt door \textcite{dubey2005recognition} gedefinieerd als: Schaalbaarheid is het vermogen om de gewenste snelheid te bieden en te onderhouden al dan niet verbeteren. De industrie heeft hier twee manieren voor. ``Scale up"\ is de methode waarbij hardware wordt vervangen zodat betere prestatie geleverd kan worden. Bij ``scale out"\ wordt er extra hardware aangesloten zodat een bestaand systeem de toenemende werkdruk kan ondersteunen.
\end{itemize}

\begin{comment}
TODO
\begin{itemize}
    \item Data aggregaties moeten plaatsvinden zodra kosten beschikbaar zijn per publisher
    
    \item voor probleem scenario 1 en 4, moet mogelijk zijn om voor een individuele webwinkels de data te aggregeren zodat er correcties kunnen worden gemaakt bij data kwaliteit issues. 
    \item de snelheid waarmee data wordt verwerkt moet snel genoeg zijn om 30 dagen aan data binnen 1 dag te verwerken.
    
    \item De tijd die het kost voor data aggregaties moet voorspelbaar zijn, dit kan worden bereikt wanneer de oplossing lineair schaalbaar is.
\end{itemize}

\end{comment}


