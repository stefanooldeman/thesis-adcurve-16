Wat zijn toonaangevende methodes en technologieën om statistieken te berekenen, die passen bij de wensen en eisen van de opdracht?

Uit verschillende websites zijn voorbeelden van technologieën gevonden voor het verwerken van data gevonden, zie tabel \ref{tab:sites}. Sommige van de technologieën zijn bekend binnen de organisatie en in overleg met de opdrachtgever zijn een aantal richtingen geselecteerd om te onderzoeken, namelijk: databases, distributed systemen, en programmeer talen die optimaal gebruik maken van hardware performance.

\begin{table}[h]
\caption{Voorbeelden van mogelijke technologieën}
\label{tab:sites}
\def\arraystretch{1.5}
\begin{tabular}{|l|p{12.5cm}|}
\hline
\textbf{Voorbeelden} & \textbf{Website}                                                                 \\ \hline
Databases     & https://github.com/onurakpolat/awesome-bigdata                                     \\ \hline
Hadoop                                  & https://experfy.com/blog/hadoop-market-size-adoption-growth-2020/                  \\ \hline
Spark                                   & http://datanami.com/2014/11/21/spark-just-passed-hadoop-popularity-web-heres/      \\ \hline
Hardware                  & http://radar.oreilly.com/2014/01/a-compelling-family-of-dsls-for-data-science.html \\ \hline
\end{tabular}
\end{table}

De huidige oplossing maak gebruik van een MapReduce algoritme. Doordat de bestaande oplossing wordt getest met behulp van een nieuwe technologie zal dit worden meegenomen in het onderzoek. Er zijn eerdere vermeldingen gevonden van MapReduce onder een andere naam, Monte Carlo is omschreven door \textcite{asanovic2006landscape} en eerdere referenties zijn gevonden in \textcite{lee2010debunking}. In het onderzoek zal ook duidelijk hoe verschillende technologieën hier anders mee om gaan.