Uit de verschillende oplossingsrichtingen zijn een aantal technologieën onderzocht in sectie \ref{sec:deelvraag2} t/m \ref{sec:hardware}. Deze zijn besproken in een groepsdiscussie waarbij de volgende non functionele eisen worden getoetst. Hiervoor is een checklist samengesteld, gebaseerd op de projecteisen omschreven in \ref{sec:deelvraag1}, en aanvullende eisen (1 en 5) die ter spraken zijn gekomen tijdens de discussie. Per vergelijkingsmatrix is af te lezen hoe iedere technologie scoort tegenover de volgende checklist in figuur \ref{tab:checklist}.

\begin{table}[h]
\centering
\caption{Checklist met criteria gebruikt in vergelijkingsmatrix}
\label{tab:checklist}
\def\arraystretch{1.2}
\begin{tabular}{l|p{11cm}c}
label       & criteria                                                          & zwaarte \\ \hline
support     & genoeg ondersteuning vanuit open source of commerciële industrie; & 3       \\
scalability & indicatoren tot schaalbaarheid en high performance;               & 2       \\
productief  & ondersteund een productief programmeer model;                     & 2       \\
testbaar    & code is testbaar door middel van unit tests.                      & 1       \\
kennis      & er is genoeg kennis beschikbaar binnen of buiten de organisatie   & 1      
\end{tabular}
\end{table}

\subsubsection{\textbf{Database oplossingen}}

In vergelijkingsmatrix \ref{tab:matrix_databases} wordt duidelijk dat de database slecht worden beoordeeld op het gebied van testbaarheid en kennis. Er is hiervoor niet genoeg ondersteuning binnen de organisatie om gebruik te maken van een database oplossing. De grootste zorg is de kans dat toenemende complexiteit van SQL queries en het gebrek aan expert kennis voor problemen zorgt.

\begin{table}[bh]
\caption{Vergelijkingsmatrix waarin eisen worden getoetst tegenover de gevonden databases}
\label{tab:matrix_databases}
\begin{tabular}{|p{3cm}|l|l|l|l|l|l|}
\hline
           & support & scalability               & productief & testbaar & kennis &     (score)  \\ \hline
GreenplumDB & +       & +                         & -          & -        & -      & 5     \\ \hline
Vertica     & +       & +                         & +          & -        & -      & 7     \\ \hline
Redshift    & +       & +                         & +          & -        & -      & 7     \\ \hline
\end{tabular}
\end{table}

\subsubsection{\textbf{Distributed oplossingen}}

In vergelijkingsmatrix \ref{tab:matrix_distributed} wordt duidelijk dat de distributed oplossingen slecht worden beoordeeld op het gebied productiviteit. De voornaamste verklaring hiervoor is ook naar voren gekomen in het onderzoek \ref{sec:distributed}. De tijd die een MapReduce job in beslag neemt is onvoorspelbaar bij adhoc queries. Dit kan betekenen dat het lang duurt voordat een developer feedback krijgt tijdens het ontwikkelen. In tegenstelling is dit geen probleem bij Spark omdat een cluster in standalone-mode op een locale laptop kan worden gestart. Daarom scoort Spark veel hoger. Over het testen van implementaties in Spark valt nog te onderzoeken. Daarnaast is het een afweging waard om code te implementeren in Python, maar uit onderzoek blijkt een implementatie in Spark SQL 10x sneller te zijn door een optimale vertaling van SQL \parencite{armbrust2015spark}.

\begin{table}[bh]
\caption{Vergelijkingsmatrix waarin eisen worden getoetst tegenover de gevonden distributed systemen}
\label{tab:matrix_distributed}
\begin{tabular}{|p{3cm}|l|l|l|l|l|l|}
\hline
                & support & scalability               & productief & testbaar & kennis &     (score)  \\ \hline
Hadoop / Hive   & +       & +                         & -          & -        & +      & 6     \\ \hline
Hadoop / Python & +       & +                         & -          & +        & +      & 7     \\ \hline
Spark SQL       & +       & +                         & +          & -        & +      & 8     \\ \hline
Spark / Python  & +       & +                        & +          & +        & +      & 8     \\ \hline
\end{tabular}
\end{table}

\clearpage

\subsubsection{\textbf{Hardware oplossingen}}

In vergelijkingsmatrix \ref{tab:matrix_hardware} wordt duidelijk dat deze programmeer talen indicatoren hebben om hoge performance te bereiken. Tijdens het onderzoek in \ref{sec:hardware} zijn er veel expert talen besproken. In vergelijkingsmatrix zijn slechts een aantal van deze talen opgenomen ter referentie, namelijk Verilog en C++ en zijn al eerder besproken als minder productief. Het onderwerp scalability moet anders worden geinterperteerd in dit geval, omdat er geen sprake is van een dataprocessing tool. Toch is de consensus dat deze talen de mogelijkheid bieden om hoge performance te behalen. Met behulp van een up-scaling methode of out-scaling door het inzetten van de aanwezige architectuur met Resqueue. Op deze manier kun data bijvoorbeeld per webshop worden verwerkt over verschillende machines, vergelijkbaar met MapReduce. De testbaarheid van programmeer talen is erg goed beoordeeld. Als laatste zijn de onderzochte DSL talen OptiSQL en Forge slecht beoordeeld door het gebrek aan beschikbare informatie, kennis en documentatie.

\begin{table}[bh]
\caption{Vergelijkingsmatrix waarin eisen worden getoetst tegenover de gevonden programmeer talen}
\label{tab:matrix_hardware}
\begin{tabular}{|p{3cm}|l|l|l|l|l|l|}
\hline
           & support & scalability               & productief & testbaar & kennis &     (score)  \\ \hline
Verilog    & -       & +                         & -          & -        & -      & 2     \\ \hline
C++        & +       & +                         & -          & -        & +      & 6     \\ \hline
OptiSQL    & -       & +                         & +          & +        & -      & 5     \\ \hline
Forge      & -       & +                         & +          & +        & -      & 5     \\ \hline
Golang     & +       & +                         & +          & +        & +      & 8     \\ \hline
\end{tabular}
\end{table}

\subsubsection{\textbf{Conclusie}}

Alle omschreven gevonden technologieën in Deelvraag 2 zijn beoordeeld. Hiermee wordt de volgende vraag beantwoord: ``Wat zijn toonaangevende methodes en technologieën om statistieken te berekenen, die passen bij de wensen en eisen van de opdracht?". Voor de beoordeling is onderzoek gedaan te lezen in hoofdstuk \ref{sec:deelvraag2}. Vervolgens zijn alle omschreven technologieën beoordeeld door middel van de checklist (zie tabel \ref{tab:checklist}) en aan de hand van groupsdiscussies.

Hieruit is gekomen dat Golang en Spark met een 8 zijn beoordeeld, daarna de databases Vertica en Redshift met een 7. Omdat er niet genoeg consensus is over het toepassen van een database oplossingen wordt deze niet getest. Dit betekend dat Golang en Spark worden getest in een tweetal Proof of Concept 's.