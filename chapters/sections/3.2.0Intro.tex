Wat zijn toonaangevende methodes en technologieën om statistieken te berekenen, die passen bij de wensen en eisen van de opdracht?

Voor data aggregaties worden veelal methodes als MapReduce gebruikt, deze methode werd eerder vernoemd als Monte Carlo door \textcite{asanovic2006landscape} en eerdere referenties zijn te vinden in \textcite{lee2010debunking}. Het algoritme wordt in de huidige situatie toegepast en is karakteristiek voor de gegeven use case. Later wordt duidelijk hoe verschillende technologieën hier anders mee om gaan. Zo wordt gekeken mogelijke hardware specifieke oplossingen in \ref{sec:hardware}.


Uit verschillende bronnen zijn richtingen gevonden voor mogelijke oplossingen. Zoals het gebruik van database oplossingen, distributed oplossingen, en hardware oplossingen. \footnote{
Uit de volgende opgevraagde webpagina's zijn verschillende onderzoeksrichtingen gevonden. De github pagina heeft over 200 technologieën en referenties naar verschillende papers.
\begin{enumerate}
    \item \url{https://github.com/onurakpolat/awesome-bigdata}
    \item \url{https://experfy.com/blog/hadoop-market-size-adoption-growth-2020/}
    \item \url{https://cs.arizona.edu/~bkmoon/papers/sigmodrec11.pdf}
    \item \url{http://datanami.com/2014/11/21/spark-just-passed-hadoop-popularity-web-heres/}
    \item \url{http://radar.oreilly.com/2014/01/a-compelling-family-of-dsls-for-data-science.html}
\end{enumerate}
}


De gevonden oplossingen zullen worden getoetst aan de volgende non-functional requirements:
\begin{enumerate}[label=(\alph*)]
   \item genoeg ondersteuning vanuit open source of commerciële industrie;
   \item indicatoren tot schaalbaarheid en high performance;
   \item ondersteund een productief programmeer model;
   \item code is testbaar;
   \item interne en externe beschikbare kennis in de vorm van documentatie en developers.
\end{enumerate}

Zie de volgende tabellen waarbij deze requirements worden getoetst: \ref{tab:matrix_databasese}, \ref{tab:matrix_distributed} en \ref{tab:matrix_hardware}

\begin{comment}
Er zijn een aantal oplossingen gevonden die maximaal gebruik kunnen maken van de performance die moderne hardware te bieden heeft
- clustered databases analytics / datawarehouses (RDBMS): google bigquery, vertica, redshift, hadoopDb, Teradata
- map reduce platform: spark, hadoop, disco etc.
- Hardware, software solutions: Forge en Golang Python, MATLAB, R, Golang, Rust
\end{comment}