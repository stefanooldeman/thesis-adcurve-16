Een mogelijke oplossing is om een Datawarehouse te implementeren waarin alle benodigde data bronnen worden beheerd en data aggregaties worden uitgevoerd.

Voor het maken van data aggregaties zijn basis SQL functies nodig: SUM, MIN, MAX, AVERAGE, samen met de SQL clauses: GROUP BY en HAVING. \parencite{data-mining}
Omdat bijna alle database systemen deze functies ondersteunen is er gekeken naar databases die een hoog volume aan data kunnen verwerken.

Veel Database Management Systemen (DBMS), ondersteunen de benodigde functies maar implementeren transactionele mechanismes om data integriteit te garanderen, bijvoorbeeld: "locking-based concurrency control". Dit introduceert significante overhead. \parencite{harizopoulos2008oltp}. Verder raakt een DBMS overbelast wanneer een gevraagde opdracht\footnote{ Ieder SQL statement wordt vertaalt naar een opdracht door middel van een query plan.} niet optimaal kan worden uitgevoerd. Zo moet de opgevraagde dataset in het geheugen (RAM) passen, of er wordt terug gevallen op de harde schijf (zogeheten "Disk spills"). De gegeven use-case vraagt om het frequent uitvoeren van aggregaties en corrigeren van data bronnen. De combinatie hiervan introduceert mogelijke performance problemen. \parencite{kersten2011researcher}
%"Moreover, database queries often contain blocking operations that lead to a pipeline stall or spilling large intermediates back to the disks." \parencite{kersten2011researcher}

% \footnote{Door frequent te aggregeren en corrigeren van data bronnen door middel van incremental loading of vervangen van datasets door 'UPDATE' of 'DELETE and INSERT' statements}
% Het plan in onze use-case is om frequent data aggregaties uit te voeren zodat statestieken in adcurve worden gecorrigeerd.

In vergelijking tot eerder genoemde OLTP (Online Transaction Processing) databases uit \textcite{kersten2011researcher} zijn er genoeg OLAP (Online Analytical Processing) systemen beschikbaar voor Data Warehouse toepassingen \parencite{data-mining}.

% In de paper "How To Build a High-Performance Data Warehouse" worden door de auteurs drie verschillende architectuur storage engines omschreven: Shared Memory, Shared Disk en Shared Nothing.

Michael Stonebraker, database pionier en origineel auteur PostgreSQL
\footnote{
 Opvallend is dat Bijna alle High Performance en Analytical databases afstammen van Postgres \parencite{postgresforks}. Dit dient als getuige van de enorme hoeveelheid research die is geïnvesteerd in (opensource) oplossingen op basis van PostgeSQL.
}
, omschrijft drie verschillende database storage engines: shared memory, shared disk en shared nothing. "Because shared nothing does not typically have nearly as severe bus or resource contention as shared-memory or shared-disk machines, shared nothing can be made to scale to hundreds or even thousands of machines. Because of this, it is generally regarded as the best-scaling architecture". \parencite{dewitt2006build}


\subsubsection{\textbf{Conclusie}}

Er zijn verschillende beschikbare technologieën gevonden in de vorm een DBMS op basis van onderzoek door \textcite{dewitt2006build}. Daarnaast is Redshift toegevoegd, een datawarehouse in de cloud, overigens ook op basis van PostgreSQL.

% Vervolgens  vergelijkings matrix zijn de requirements vergeleken met de eigenschappen van iedere DBMS.

\begin{itemize}
    \item Teradata
    \item GreenplumDB, door Pivotal
    \item HP Vertica
    \item Amazon Redshift
\end{itemize}



% De volgende lijst van database systemen bieden ook de mogelijkheid tot Analytics met behulp van "windowing" of "time-series" functies.
% Redshift
% Vertica
