
De volgende lijst van database systemen beschikken over analytical functionaliteiten nodig voor het aggregeren en creeëren van statistieken. Voor het maken van data aggregaties zijn baisis SQL functies nodig: SUM, MIN, MAX, AVERAGE, samen met de SQL clauses: GROUP BY en HAVING. \parencite{data-mining} Omdat bijna alle database systemen deze functies ondersteunen is er gekeken naar databases die een hoog volume aan data kunnen verwerken. 
Omdat er nu per dag zo'n 1.8Gb aan data voor alleen visits wordt verwerkt en dit extropaleert naar (1.8/480) * 100000 = 375Gb per dag.

Traditionele databases systemen zijn gemaakt om transactionele queries te beantwoorden. Deze zullen minder schaalbaar zijn doordat er verschillende fases zijn die overhead introduceren. \parencite{harizopoulos2008oltp}
Daarnaast introduceert het frequent updaten en corrigeren data bronnen mogelijke performance problemen doordat data consistentie wordt gegarandeerd met methodes als "table locks". Verder raken databases overbelast wanneer de opgravraagde database query resulteert in een dataset die nit in het geheugen is te plaatsen.  daarom tot een trager, overbelast systeem. \parencite{kersten2011researcher}

Database queries kunnen overigens erg traag raken wanneer databases groeien en query resultaten niet meer in RAM passen. Hierbij wordt resulteert een terugval op de hardeschijf (zogeheten "Disk spills") in het vertragen van een datapipeline. Ook zijn er verschillende anakedotes vanuit de organisatie waarbij databases zoals MySQL sporadisch, overbelast raakt.

%"Moreover, database queries often contain blocking operations that lead to a pipeline stall or spilling large intermediates back to the disks." \parencite{kersten2011researcher}


Het plan in onze use-case is om frequent data aggregaties uit te voeren zodat statestieken in adcurve worden gecorrigeerd. Dit kan er uiteindelijk voor zorgen dat locking systemen om data consistentie te garanderen een bottle neck op a 
Traditionele databases garanderen waarbij een transactionele niet worden beschouwd als een verkiesbare technologie. Daarom is gekeken naar databases met een andere storage engine.

Opvallend is dat bijna alle distributed en analytics databases van PostgreSQL afstammen \parencite{postgresforks}

Hieruit valt te concluderen dat de business intelligence industry veel research heeft geinvesteerd in op basis van Postgress technology.

Michael Stonebraker is origineel auteur Postgress en later vele high performance databases. In zijn paper "How To Build a High-Performance Data Warehouse"\  omschrijft hij drie verschillende architectuur typen / storage engines: Shared Memory, Shared Disk en Shared Nothing.
"Because shared nothing does not typically have nearly as severe bus or resource contention as shared-memory or shared-disk machines, shared nothing can be made to scale to hundreds or even thousands
of machines. Because of this, it is generally regarded as the best-scaling architecture". \parencite{dewitt2006build}



In de paper worden de volgende databases genoemd:
\begin{table}[bh]
\centering
\caption{My caption}
\label{my-label}
\begin{tabular}{|l|l|l|}
\hline
\textbf{Shared memory} & \textbf{Shared Disk} & \textbf{Shared Nothing} \\ \hline
Microsoft SQL Server   & Oracle RAC           & Teradata \\
\hline
PostgresSQL            & Sysbase IQ           & Netezza \\
\hline
MySQL                  &                      & IBM DB2 \\
\hline
                       &                      & GreenplumDB \\
\hline
                       &                      & Vertica \\
\hline
\end{tabular}
\end{table}

\subsubsection{\textbf{Conclusie}}

TODO: Vorm een conclusie over databases die bij de requirements passen.