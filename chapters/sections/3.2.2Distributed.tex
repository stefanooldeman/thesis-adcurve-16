In 2009 werd geconstateerd door \cite{aarnio2009parallel} werd MapReduce als geidentificeerd als een "Prominent parallel data processing tool" dat tractie kreeg vanuit zowel de industrie als de academische wereld. Zo concludeerde zij dat: "Naast traditionele Database Management Systems is Map Reduce een zeer schaalbaar en flexibele oplossing is voor een variatie aan data analyses" \parencite{aarnio2009parallel} \\

Voor langere tijd nu, staat Hadoop bekende als een volwasse technologie en implementatie van Map Reduce, zo schrijft Gartner in een recent rapport: "The open-source software framework known as Apache Hadoop has gained sizable acceptance in organizations, spurred on by the growing digital business appetite for big data" \parencite{hadoop2013selection}. \\


"Joe Hellerstein, our local expert in databases, said the future of databases was large data
collections typically found on the Internet. A key primitive to explore such collections is
MapReduce, developed and widely used at Google \parencite{dean2008mapreduce}" \parencite{asanovic2006landscape}, 

Kort nadat UC Berkley zijn cluster framework "Spark" heeft gedoneerd aan de Apache foundation in 2014, heeft het project enorme aandacht ontvangen vanuit de hele industrie. \parencite{spark2015intro}. Zo is er ook vanuit Shop2market enorme interesse in Apache Spark. Er zijn al eerder in 2015 Proof of Concept uitgevoerd met betrekking tot machine learning applicaties. \\

Terwijl nog veel organisaties zweven tussen het begrijpen en het vinden van de toepassing voor Hadoop \parencite{hadoop2015adoption}, is sinds 2013 een alternatief in memory implementatie van o.a. Map reduce enorm populair  \parencite{spark2015intro}

\subsubsection{Conclusie}

De uiteindelijke selectie voor dit onderzoek is beperkt. De frameworks die zijn geselecteerd hebben genoeg herkenning vanuit de organisatie en industrie. De enorme lijst op github is getuige van de enorme hype rondom Big Data. Toch zijn veel organisaties nog zoekend naar wat technologie betekent en hoe het waarde toevoegd\parencite{hadoop2015adoption}.

Verder zijn er in de eerder genoemde lijst (github) enorm veel tools benoemd. In dit onderzoek zullen naast Hadoop en Naast eerder besproken frameworks Spark en Hadoop zijn er andere technologieën waar colleg's eerdere ervaring mee hebben.

De selectie van technologieën komt uit op

\begin{itemize}
    \item Disco
    \item Apache Flink
    \item Hadoop (Apache MapReduce)
    \item Apache Spark
\end{itemize}

Streaming methodes zoals bijvoorbeerld Apache Storm zijn worden als toonaangeevend gezien maar zijn niet geselecteerd. Dit omdat de vorm waarin de data beschikbaar wordt gesteld aan het aggregatie proces altijd in batch is. Het valt buiten de scope om deze data flows te veranderen.