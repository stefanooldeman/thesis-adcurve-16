% Wanneer grootte hoeveelheden data worden verwerkt is een goed alternatief "Joe Hellerstein, our local expert in databases, said the future of databases was large data collections typically found on the Internet. A key primitive to explore such collections is MapReduce" \parencite{asanovic2006landscape}, 


- voor het verwerken van grootte hoevelheden data
- toepassingen / reden van map reduce
- toepassingen / reden van Spark
- andere ervaringen

Verdere mogelijke technologieën zijn alternatieven 
De huidige oplossing maakt gebruik van MongoDB die een MapReduce framework heeft geimplementeerd.


Doordat de verhouding tussen hardware capiciteit en de kosten hiervan steeds gunstiger werden en de toenemende hoeveelheid data die gegenereerd wordt door het internet..

Voor het verwerken van de toenemende hoeveelheid data gegenereerd door  internet
"Joe Hellerstein, our local expert in databases, said the future of databases was large data
collections typically found on the Internet. A key primitive to explore such collections is
MapReduce, developed and widely used at Google \parencite{dean2008mapreduce}" \parencite{asanovic2006landscape}, 

Het alternatief naast databases zijn MapReduce frameworks, waarbij data wordt opgeslagen en beheerd in een cluster, aldus een distributed systeem. Veel van de populaire 

Voor het verwerken van grootte hoeveelheden data werden distributed systemen bij Google succesvol ingezet, 
Skeptisch, databases niet geschikt voor het verwerken van grootte hoeveelheden data. 

Distributed systemen zijn in 2006 geintroduceerd door google met het GFS en MapReduce. \parencite{dean2008mapreduce}
In \textcite{asanovic2006landscape} werd MapReduce nog als het alternatief gezien voor het verwerken van grootte datasets. En hoewel databases later ook distributed systemen werden (Shared nothing, column stores zijn nl distributed), en dit tot Petabyte schaal kan worden ingezet. \parencite{lamb2012vertica}
Is het vele malen duurder!
De populaire MapReduce platformen zijn nog altijd te configureren met "Commodity Hardware" etc. \parencite{dean2008mapreduce}


Voor langere tijd nu, staat Hadoop bekende als een volwasse technologie en implementatie van Map Reduce, zo schrijft Gartner in een recent rapport: "The open-source software framework known as Apache Hadoop has gained sizable acceptance in organizations, spurred on by the growing digital business appetite for big data" \parencite{hadoop2013selection}.

Vele organisaties weten echter nog niet hoe ze deze technologie effectief kunnen inzetten. OOk al hebben organisaties nu wel de kennis, blijft de adoptie nog altijd laag. \parencite{hadoop2015adoption}


\begin{comment}
De evolutie van distributed applictions Google File System -> Map Reduce -> Apache Hadoop -> Hype is groot maar Mainstream Adoption is nog erg laag, complex landschap. De organisatie heeft eerder een Proof Of Concept gerealiseerd maar zag op tegen het beheer van een cluster. 
Tegenwoordig is er ook success met hadoop in de cloud, aangeboden door Amazon EMR.
Minpunten:
- Hoge kosten
- Opstart tijd van 17 minuten (zie benchmarks Spark vs Hadoop)

Alternatief Spark erg enthasiaust vanuit het development Team, ook mogelijk op Amazon EMR
\end{comment}

% Rond 2009 MapReduce werd geïdentificeerd als een "Prominent parallel data processing tool" dat tractie kreeg vanuit zowel de industrie als de academische wereld. Zo concludeerde zij dat: "Naast traditionele Database Management Systems is Map Reduce een zeer schaalbare en flexibele oplossing is voor een variatie aan data analyses". \parencite{aarnio2009parallel} 


% Kort nadat UC Berkley zijn cluster framework "Spark" heeft gedoneerd aan de Apache foundation in 2014, heeft het project enorme aandacht ontvangen vanuit de hele industrie. \parencite{spark2015intro}. Zo is er ook vanuit Shop2market enorme interesse in Apache Spark. Er zijn al eerder in 2015 Proof of Concept uitgevoerd met betrekking tot machine learning applicaties.

% Terwijl nog veel organisaties zweven tussen het begrijpen en het vinden van de toepassing voor Hadoop \parencite{hadoop2015adoption}, is sinds 2013 een alternatief in memory implementatie van o.a. Map reduce enorm populair  \parencite{spark2015intro}

\subsubsection{\textbf{Conclusie}}

De uiteindelijke selectie voor dit onderzoek is beperkt. De frameworks die zijn geselecteerd hebben genoeg herkenning vanuit de organisatie en industrie. De enorme lijst op github is getuige van de enorme hype rondom Big Data. Toch zijn veel organisaties nog zoekend naar wat technologie betekent en hoe het waarde toevoegd\parencite{hadoop2015adoption}.

Verder zijn er in de eerder genoemde lijst (github) enorm veel tools benoemd. In dit onderzoek zullen naast Hadoop en Naast eerder besproken frameworks Spark en Hadoop zijn er andere technologieën waar colleg's eerdere ervaring mee hebben.

De selectie van technologieën komt uit op

\begin{itemize}
    \item Disco
    \item Apache Flink
    \item Hadoop (Apache MapReduce)
    \item Apache Spark
\end{itemize}

Streaming methodes zoals bijvoorbeerld Apache Storm worden als toonaangeevend gezien maar zijn niet geselecteerd. Dit omdat de vorm waarin de data beschikbaar wordt gesteld aan het aggregatie proces altijd in batch is. Het valt buiten de scope om deze data flows te veranderen.