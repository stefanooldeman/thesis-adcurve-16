Een van de oplossingen die wordt onderzocht is het gebruik maken van krachtige hardware componenten (zowel de CPU en de GPU). Dit is mogelijk met expert talen zoals; Verilog, CUDA of OpenCL of C++.

Op de universiteit van Standford is hier veel onderzoek gedaan. Veel hardware specifieke kennis moet in code worden uitgedrukt: "programming these devices to run efficiently and correctly is difficult, error-prone, and results in software that is harder to read and maintain." \parencite{sujeeth2011optiml}. Daarom zijn er een serie aan DSL's ontwikkeld\footnote{Veel voorkomende voorbeelden van DLS's zijn HTML, SQL en \TeX \parencite{sigplan2000dsl}}

Terwijl \textcite{sujeeth2011optiml} een veelbelovende oplossing lijkt aan te bieden is het project\footnote{Zie de website van OptiML: \url{https://stanford-ppl.github.io/Delite/optiml/index.html}} nog altijd in Alpha versie. Het zelfde team presenteerde in 2014 Forge gepresenteerd. Een DSL voor expert talen waar de test resultaten erg veelbelovend zijn. "Forge-generated Delite DSLs perform within 2x of hand-optimized C++ and up to 40x better than Spark, an alternative high-level distributed programming environment." \parencite{sujeeth2014forge}. Helaas lijkt hierbij hetzelfde probleem te spelen en is er niet genoeg documentatie gevonden om hier een POC mee te starten.

Binnen Shop2market zijn veel performance gerelateerde projecten opgelost met Golang. De taal is modern heeft een alternatief concurrency model. "Concurrency is important to the modern computing environment with its multicore machines running web servers with multiple clients, what might be called the typical Google program. This kind of software is not especially well served by C++ or Java, which lack sufficient concurrency support at the language level." \parencite{pike2012go}

\subsubsection{\textbf{Conclusie}}

De besproken technologieën zijn de DSL talen: OptiML, OptiSQL en Forge. Bij gebrek aan praktische voorbeelden en algemeene documentatie is er voor gekozen om hier niet verder in te investeren. Er zou niet voldoende kennis zijn om een mogelijke POC naar productie te brengen. Dit is teleurstellend aangezien de veelbelovende benchmarks.

Echter wordt niet aangegeven welke algoritmes er zijn gebruikt tijdens deze benchmarks. In het geval van data aggregatie\footnote{Voor data aggregaties worden methodes als MapReduce of Monte Carlo gebruikt} wordt er mogelijk een betere performance behaalt door het gebruik van multicore CPU in plaats van GPU processor capiciteit. Zo omschrijft \textcite{lee2010debunking} "Monte Carlo algorithms are generally compute-bound with regular access patterns, which makes it a very good fit for SIMD architectures".

Gegeven de huidige requirements en bevindingen geeft dit genoeg motivatie om een POC te starten in Golang.

